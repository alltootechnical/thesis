\section{Introduction}
Digital image security is a growing concern, and various image encryption schemes have been developed and analyzed in the literature~\cite{murugan_survey_2018, jain_image_2016, khoirom_cryptanalysis_2018}. Cryptosystems allow for the secure transmission of data by way of encryption and decryption schemes. 
Of particular interest is homomorphic image encryption, which allows operations to be performed on encrypted images. These operations are then preserved when the images is decrypted. In this manner, images are manipulated without compromising data privacy, as the image being operated on remains secure~\cite{fontaine_survey_2007, sen_homomorphic_2013}. 

In the Philippine setting, preservation of data privacy is of utmost importance when it comes to outsourcing work involving sensitive data to local companies with the ratification of the Data Privacy Act of 2012. Thus recent developments in applications of homomorphic encryption on non-text-based data (such as images) are becoming relevant today.

In the area of homomorphic image encryption, there is interest in being able to perform image manipulation operations such as image adjustment, image filtering, and morphological extraction operations~\cite{ziad_cryptoimg:_2016, gonzalez_digital_2008}. There is also work on facial recognition on encrypted images~\cite{turk_eigenfaces_1991, hutchison_privacy-preserving_2009}. While secure linear image operations have been implemented~\cite{ziad_cryptoimg:_2016}, preprocessing which requires non-linear operations, such as those in~\cite{oravec_illumination_2010}, has yet to be performed in the encrypted domain.

Homomorphic encryption schemes have two main limitations: first, they are limited in the operations that can be performed on encrypted data efficiently~\cite{li_elliptic_2012}, and second, they are slower than non-homomorphic schemes~\cite{sen_homomorphic_2013}. This paper presents implementations of non-linear image operations under a homomorphic encryption scheme, and evaluates their applicability and performance.

To investigate the applicability of homomorphic image manipulation, in Section \ref{sec:chapter_2}, we introduce three homomorphic cryptosystems prominent in the literature: the Paillier, Damg{\aa}rd--Geisler--Kr{\o}igaard (DGK), and Brakerski--Gentry--Vaikuntanathan (BGV) cryptosystems, as well as relevant past implementations. In Section \ref{sec:chapter_3}, we implement a software library allowing for common linear and non-linear image processing operations using each of the three cryptosystems, assuming a client-server model for secure computation.

To evaluate the performance of homomorphic image manipulation, in Section \ref{sec:chapter_4}, we test the time efficiency and accuracy of common image processing operations under homomorphic encryption. We test for image quality of the resulting images using mean squared error (MSE), peak signal to noise ratio (PSNR), and structural similarity index (SSIM), which are measures of image quality used for assessing image encryption schemes~\cite{ahmed_benchmark_2016}.

% Training and test data are taken from the publicly available \texttt{faces94} dataset (\url{https://cswww.essex.ac.uk/mv/allfaces/faces94.html}) maintained by the University of Essex.

% \subsection{Significance of the Study}
% This study aims to directly address one of the current problems in the research of homomorphic cryptosystems: the practicality of homomorphic encryption~\cite{sen_homomorphic_2013}. The literature shows existing homomorphic cryptosystems and how they support primitive operations (usually addition and multiplication) on encrypted data, however, for homomorphic cryptosystems to be useful for practical use, efficient support for more complicated image processing operation must be demonstrated as well. While implementations of image processing operations in a homomorphic cryptosystem exist in the literature~\cite{ziad_cryptoimg:_2016, garay_algorithms_2014}, our study targets the more computationally intensive task of facial recognition and detection.



% We wish to contribute to existing models for the secure transmission and modification of image data by developing a software library which allows the use and comparison of various homomorphic encryption algorithms. Our software library would allow service providers to perform image manipulation for clients without compromising data privacy.
