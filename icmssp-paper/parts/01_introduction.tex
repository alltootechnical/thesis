\section{Introduction}
% \subsection{Context of the Study}
% Digital data privacy and security is a growing concern in various fields, such as cloud computing~\cite{potey_homomorphic_2016}, health information systems~\cite{kester_cryptographic_2015}, and video surveillance~\cite{upmanyu_efficient_2009}. 
Digital image encryption is a growing concern today, with many algorithms existing in the literature~\cite{murugan_survey_2018}.  

% TODO : Mention the four HE schemes here
Of particular interest is homomorphic encryption, where operations such as addition and multiplication can be performed on encrypted data. These operations are then preserved when the data is decrypted. In this manner, data is manipulated without compromising data privacy, as the encrypted data being operated on remains secure~\cite{fontaine_survey_2007, sen_homomorphic_2013}. In the area of homomorphic image encryption, there is interest in being able to perform image manipulation operations on the encrypted data such as image adjustment, filtering, and morphological/feature extraction operations~\cite{ziad_cryptoimg:_2016, gonzalez_digital_2008}. More advanced operations have also been implemented using homomorphic cryptosystems, such as facial recognition \cite{hutchison_privacy-preserving_2009} and neural networks \cite{hesamifard_cryptodl:_2017}. This study in particular focuses on facial recognition.

Our main research objective is to assess the practicality of various homomorphic cryptosystems with regard to image processing on faces.
% Homomorphic cryptosystems are known to be slower in terms of encryption, operation, and decryption time~\cite{sen_homomorphic_2013}. Furthermore, the operations on encrypted data supported by homomorphic cryptosystems are limited, and some schemes require preprocessing to be performed on images, limiting the range of operations which can be performed efficiently~\cite{li_elliptic_2012}.

% TODO: be specific here, integrate objectives and scope
% Focus objectives on new developments
In this study, we accomplish the following:
\begin{enumerate}
    \item Implement common linear and non-linear image processing operations using the Paillier, DGK, DGHV, and BGV homomorphic encryption schemes.
    \item Compare the time efficiency and accuracy of common image processing operations under these homomorphic encryption schemes, and of common image processing operations without using homomorphic encryption;
    \item Create a library to allow for convenient use and comparison of these homomorphic encryption schemes for image processing and facial recognition.
\end{enumerate}
We will implement and test the following categories of image manipulation operations:
\begin{enumerate}
	\item Non-linear image intensity adjustment (logarithm and power-log transformations)
	\item Facial detection and recognition using the eigenface approach \cite{turk_eigenfaces_1991}.
\end{enumerate}
For this study, we adopted the benchmark for evaluating image quality of image encryption schemes presented by Ahmed, et. al~\cite{ahmed_benchmark_2016}. The presented benchmark reflects many tests used in other literature~\cite{ahmad_efficiency_2012, wu_npcr_2011}.
\begin{enumerate}
    \item Mean Squared Error (MSE)
    \item Peak Signal to Noise Ratio (PSNR)
    \item Structural Similarity Index (SSIM)
\end{enumerate}
A detailed overview of the above will be provided in the methodology. Grayscale images will be used in the study.
A client-server model for secure computation will be assumed 

% Training and test data are taken from the publicly available \texttt{faces94} dataset (\url{https://cswww.essex.ac.uk/mv/allfaces/faces94.html}) maintained by the University of Essex.

% TODO: shorten this
% \subsection{Significance of the Study}
This study aims to directly address one of the current problems in the research of homomorphic cryptosystems: the practicality of homomorphic encryption~\cite{sen_homomorphic_2013}. The literature shows existing homomorphic cryptosystems and how they support primitive operations (usually addition and multiplication) on encrypted data, however, for homomorphic cryptosystems to be useful for practical use, efficient support for more complicated image processing operation must be demonstrated as well. While implementations of image processing operations in a homomorphic cryptosystem exist in the literature~\cite{ziad_cryptoimg:_2016, garay_algorithms_2014}, our study targets the more computationally intensive task of facial recognition and detection.

In the Philippine setting, preservation of data privacy is of utmost importance when it comes to outsourcing work involving sensitive data to local companies with the ratification of the Data Privacy Act of 2012. Thus recent developments in applications of homomorphic encryption on non-text-based data (such as images) are becoming relevant today.

We wish to contribute to existing models for the secure transmission and modification of image data by developing a software library which allows the use and comparison of various homomorphic encryption algorithms. Our software library would allow service providers to perform image manipulation for clients without compromising data privacy.
