\section{Conclusion}
In this paper, we have assessed the feasibility and practicality of various homomorphic cryptosystems with regard to secure processing and manipulation of facial image data through creation of a software library that facilitates image processing operations within the encrypted domain. We have also extended the Paillier and DGK cryptosystems to support limited floating-point arithmetic, demonstrated in the implementation of the logarithm and power-law image intensity transformations. 

Preliminary results have shown that both Paillier and DGK apply image negation accurately and efficiently, as evidenced by the perfect zero MSE and infinite PSNR. For the logarithm transformation, Paillier was faster and accurate enough (with $\text{MSE} \le 30$ and $\text{SSIM} \approx 0.99$) which was rather unexpected. However both Paillier and DGK fail to produce accurate results in applying the power-law transformation, with the latter failing to produce discernible results. Based from these results, the Paillier cryptosystem is consistent enough to produce reasonable to accurate results.

\subsection{Ongoing and Future Work}
Testing of intensity transformations under the BGV cryptosystem will be started once the Pyfhel library is deemed stable enough to use, or until a similar library becomes available.
One of the authors of this paper has already informed the developers of the Pyfhel library about a bug involving exponentiation of two real numbers.

Efforts are ongoing with regard to the facial recognition tests. Our own implementation of the privacy-preserving eigenfaces by Erkin, et al. does not yet use the DGK cryptosystem for secure integer comparison in the match finding step.
Another promising image operation to consider is \textit{facial detection} under a homomorphic cryptosystem. Work has been started in order to assess the feasibility of using Haar cascades in a privacy-preserving manner.

Future work would involve improving and optimizing our implementation of non-linear intensity transformations in terms of number of operations, numerical accuracy and stability, and time efficiency. An implementation of a client-server system dealing with secure image processing can be considered.
