\section{Non-Linear Intensity Transformations}
We now describe how we are to implement the logarithm and power-law transformations.
\subsection{Logarithm Transformation}
The logarithm transformation of a pixel intensity value $x$ is defined as
\begin{align}
	T\left(x\right) = c \log\left(1 + x\right)
\end{align}
where $c$ is a constant.

In order to perform this transformation under a homomorphic cryptosystem, we must provide an approximation for $\log\left(1 + x\right)$ in terms of addition and multiplication operations (or their inverses). We have derived a closed form approximation in Appendix \ref{sec:logapproximation}.
\begin{align}
	\label{eq:scaledquadraturech3}
  \begin{split}
    &\log(1+x) \\
    &=\frac{137x^5 + 33185x^4 + 931370x^3 - 13403630x^2 - 289469315x - 713567363}
    {30(x^5 + 505x^4 + 42010x^3 + 923010x^2 + 5722005x + 8040501)} + \log{20}
  \end{split}
\end{align}

\subsection{Power-Law Transformation}
The power-law transformation of a pixel intensity value $x$ is defined as
\begin{equation}
    T\left(x\right) = cx^{\gamma}
\end{equation}
where $c>0$ and $\gamma > 0$.

Similar to the logarithm transformation, we must provide an approximation for $x^\gamma$.
We have derived an infinite series in Appendix \ref{sec:logapproximation}.
\begin{align*}
	x^\gamma &= \sum_{n=0}^{\infty}{\frac{(\gamma\log{x})^n}{n!}}\\
\end{align*}
Partial sums of the above infinite series can be calculated based on the closed form approximation for the logarithm in Equation \ref{eq:scaledquadraturech3}. 
