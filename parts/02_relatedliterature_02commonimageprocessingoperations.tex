\section{Common Image Operations}
In this study, we wish to see how the primitive operations of addition and multiplication supported by homomorphic cryptosystems can be applied to image processing operations. These operations are performed on digital images. WE can represent a digital image $A$ as an $M \times N$ matrix of pixel intensity values, each value in the range $\left[0, L-1\right]$, for some positive integer $L$. There are two types of basic operations in image processing, intensity transformations, which map an intensity value to another, spatial filters which assist in operations such as edge detection and image blurring.

\subsection{Intensity Transformation}
To define an intensity transformation on an image $A$, we define a function $T$ which maps a pixel value $r$ to a new value $r^\prime$, which we can write as $r^\prime = T\left(r\right)$. This function is then applied to every pixel in $A$. Examples of intensity transformations are image negation, log transformation, and power-law transformation.

Image negation is an example of an intensity transformation, where the resulting image would be similar to a photographic negative~\cite{gonzalez_digital_2008}. An image negation operation is defined as:
\begin{equation}
    T\left(r\right) = L-1-r
\end{equation}

The log transformation is used to enhance dark pixels or increase the dark details of an image by mapping low intensity values to a wider range of values~\cite{gonzalez_digital_2008}. This has the general form
\begin{equation}
    T\left(r\right) = c \log\left(1 + r\right)
\end{equation}
where $c$ is a constant and $r \ge 0$.

The power-law transformation is a family of transformations that have the form
\begin{equation}
    T\left(r\right) = c r^{\gamma}
\end{equation}
where $c>0$ and $\gamma > 0$. This is especially useful since many image capture and output devices such as cameras, printers and displays follow a similar power law to relate physical percieved light intensity and digital pixel intensity values. A power-law transformation defined by the above equation can calibrate the operation on these devices in a process called \textit{gamma correction}. This ensures reproducibility and accuracy of images being displayed~\cite{gonzalez_digital_2008}.

\subsection{Edge Detection and Spatial Filtering}
Edge detection is used to find and determine the boundaries in an image, commonly used in applications such as image segmentation and feature extraction. This works by detecting so-called \textit{edges}, areas that have abrupt changes in intensity.

Edge detection is usually done by using gradient operators that detect such abrupt changes. These operators are commonly known as \textit{spatial filter}, which are usually of $3 \times 3$ size. A common example of spatial filters is the Sobel operator, with two matrices (also called as kernels) $g_x$ and $g_y$ representing the horizontal and vertical components respectively.
\begin{equation}
    g_x =
    \begin{bmatrix}
        -1 & 0 & 1 \\
        -2 & 0 & 2 \\
        -1 & 0 & 1
    \end{bmatrix}
    \qquad\text{and}\qquad
    g_y =
    \begin{bmatrix}
        1 & 2 & 1 \\
        0 & 0 & 0 \\
        -1 & -2 & -1
    \end{bmatrix}
\end{equation}
To get the resulting image $I^\prime$, a convolution is performed between the original image $I$ of size $M \times N$ and the kernel $k$ of size $m \times n$. Now suppose that the pixel value of an image at point $\left(i,j\right)$ is $r_{i,j}$. Then, a transformation using spatial filters can be described as follows:
\begin{align}
    T\left(r_{i,j}\right) &= \left[k * I\right]\left(\left\lfloor\frac{m}{2}\right\rfloor, \left\lfloor\frac{n}{2}\right\rfloor \right) \\
                         &= \sum_{u=1}^{m} \sum_{v=1}^{n} \left[k_{i,j} r_{i+u, j+v} \right]
\end{align}

Spatial filters are not only used for edge detection, but there are also filters that do image smoothing (such as Gaussian blur and box blur, $b_g$ and $b$ respectively in Equation~\ref{eqn:smooth-filters}) and image sharpening, to name a few \cite{gonzalez_digital_2008}.
\begin{equation}
    \label{eqn:smooth-filters}
    b_g = \frac{1}{16}
    \begin{bmatrix}
        1 & 2 & 1 \\
        2 & 4 & 2 \\
        1 & 2 & 1
    \end{bmatrix}
    \qquad
    b = \frac{1}{9}
    \begin{bmatrix}
        1 & 1 & 1 \\
        1 & 1 & 1 \\
        1 & 1 & 1
    \end{bmatrix}
\end{equation}
% Should I put a table of other kernels too?
