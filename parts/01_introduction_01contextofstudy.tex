
\section{Context of the Study}

Digital data privacy and security is a growing concern in various fields, such as  cloud computing \cite{potey_homomorphic_2016}, health information systems \cite{kester_cryptographic_2015}, and video surveillance \cite{upmanyu_efficient_2009}. To meet these needs, various cryptosystems have been developed. A cryptosystem is a system which operates on plaintexts, ciphertexts and keys. a cryptosystem also has an encryption algorithm, which maps a plaintext to a ciphertext, and a decryption algorithm, which maps a ciphertext to a plaintext, given an appropiate key \cite[p.119]{tilborg_encyclopedia_2005}. The encryption and decryption algorithms are chosen such that the plaintext cannot easily be recovered from the ciphertext without knowledge of the key. This allows data to be securely transmitted over a insecure channel:  and thus much research has been done on the development and application of various cryptosystems.

One such particular application of cryptography is image encryption. While cryptosystems used to encrypt digital data in general, such as the Advanced Encryption System (AES) or elliptic curve cryptography, can also be used to encrypt images \cite{jain_image_2016, singh_image_2015} other algorithms created specifically for image encryption have also been developed \cite{murugan_survey_2018}.

However, research has also considered a type of cryptosystem, homomorphic cryptosystems, where in addition to allowing the secure transmission of data, it is also possible to perform computations with encrypted data, the most basic of which being addition and multiplication. Numerous homomorphic cryptosystems exist, which allow for data to be manipulated without compromising data privacy \cite{fontaine_survey_2007, sen_homomorphic_2013}. In homomorphic image encryption, there is additional interest in being able to perform image manipulation operations on the encrypted data such as image adjustment, filtering, and morphological/feature extraction operations \cite{ziad_cryptoimg:_2016, gonzalez_digital_2008}.
