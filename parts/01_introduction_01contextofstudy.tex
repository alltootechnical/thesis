
\section{Context of the Study}

Digital data privacy and security is a growing concern in various fields, such as cloud computing~\cite{potey_homomorphic_2016}, health information systems~\cite{kester_cryptographic_2015}, and video surveillance~\cite{upmanyu_efficient_2009}. To meet these needs, various cryptosystems have been developed. Cryptosystems allow plaintexts, data which is commonly understood, to be encrypted as ciphertext, data which can only be understood by authorized parties using a key~\cite{bauer_cryptosystem_2005}. The encryption and decryption algorithms are chosen such that the plaintext cannot easily be recovered from the ciphertext without knowledge of the key. This allows data to be securely transmitted over an insecure channel.

One particular application of cryptography is image encryption. While cryptosystems used to encrypt general digital data, such as the Advanced Encryption Standard (AES) or elliptic curve cryptography, can also be used to encrypt images~\cite{jain_image_2016, singh_image_2015} other algorithms created specifically for image encryption have also been developed~\cite{murugan_survey_2018}.

 Using a homomorphic cryptosystem, operations can be performed on encrypted data while keeping both the operands and resulting value secure. The result of the operations can be recovered when the data is decrypted.. In this manner, data is manipulated without compromising data privacy~\cite{fontaine_survey_2007, sen_homomorphic_2013}. While numerous homomorphic cryptosystems exist for use with general digital data, in the area of homomorphic image encryption, there is additional interest in being able to perform image manipulation operations on the encrypted data such as image adjustment, filtering, and morphological/feature extraction operations~\cite{ziad_cryptoimg:_2016, gonzalez_digital_2008}. More advanced operations have also been implemented using homomorphic cryptosystems, such as facial recognition \cite{hutchison_privacy-preserving_2009} and neural networks \cite{hesamifard_cryptodl:_2017}. This study in particular focuses on facial recognition.
