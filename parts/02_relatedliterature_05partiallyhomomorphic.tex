\section{Partially Homomorphic Cryptosystems}

In this section, we will present two classical partially homomorphic encryption (PHE) schemes which we will be testing in the study: the Paillier cryptosystem and the DGK cryptosystem. Both of these cryptosystems encrypt integer plaintexts, are additively homomorphic, and allow for the multiplication of a plaintext scalar. We will then show how these PHE schemes may be adapted for secure floating-point computation using two-party protocols. 

\subsection{The Paillier Cryptosystem}
The Paillier cryptosystem \cite{stern_public-key_1999}, developed by Pascal Paillier, is a probabilistic encryption scheme which is based on the composite residuosity class problem.

\subsubsection{Key Generation}
We choose two large primes $p$ and $q$, and set $n = pq $ and $ \lambda = \mathrm{lcm}\left(p-1,q-1\right)$.
We then define $L\left(x\right)$ to be the largest integer $v$ greater than zero such that $x-1 \geq vn$.
Then we select an integer $g$ such that $\gcd\left(L\left(g^\lambda \bmod n^2\right), n\right) = 1$ and $0\leq g \leq n^2$.
We define the public key as $(g,n)$ and the private key as $(p,q)$.

\subsubsection{Encryption and Decryption}
The encryption function to encrypt a plaintext $m \in \mathbb{Z}_n$ given a public key $(g,n)$ is defined as
\begin{align*}
  E(m) = g^m \cdot r^n \mod{n^2},
\end{align*}
where $r$ is a random non-negative integer less than $n^2$.

The decryption function to decrypt a ciphertext $c \in \mathbb{Z}_{n^2}$ given a private key $(p,q)$ is defined as
\begin{align*}
  D(c) = L(c^\lambda \bmod n^2) \times (L(g^\lambda \bmod n^2))^{-1} \mod n.
\end{align*}

\subsubsection{Homomorphic Properties of the Paillier Cryptosystem}
The Paillier cryptosystem supports additive homomorphism as well as the multiplication of a plaintext scalar to an encrypted message. These operations are defined as follows.
For all $m_1,m_2 \in \mathbb{Z}_n$ and $k\in \mathbb{N}$, the following homomorphic properties hold.
\begin{align*}
  D(E(m_1)g^k\bmod n^2)=(m_1+k)\bmod n & \text{ (add a plaintext constant)}\\
  D(E(m_1)E(m_2)\bmod n^2)=(m_1+m_2)\bmod n & \text{ (ciphertext addition)}\\
  D(E(m_1)^k\bmod n^2)= km_1\bmod n & \text{ (multiply a plaintext constant)}
\end{align*}

\subsection{The DGK Cryptosystem}
The DGK cryptosystem was published by Damg{\aa}rd, Geisler, and Kr{\o}igaard in 2007 in an effort to create a secure integer comparison scheme \cite{pieprzyk_efficient_2007, cryptoeprint:2008:321} which is widely used in the literature \cite{veugen_improving_2012}.

\subsubsection{Key Generation}
We denote $k,t,\ell$ as security parameters of the scheme, where $k>t>\ell$.
Let $p,q$ be primes such that
we can choose two $t$-bit primes $v_p$ and $v_q$ such that $v_p | (p-1)$ and $v_q | (q-1)$, and a small prime $u$ such that $u | (p-1)$ and $u | (q-1)$.
We denote $n = pq$.
We choose $g$ to be an integer of order $uv_pv_q$ and $h$ to be of order $v_pv_q$.

The DGK cryptosystem encrypts plaintexts in $\mathbb{Z}_u$ to ciphertexts in $\mathbb{Z}_n^\ast$.

The public key is $(n,g,h,u)$ and the private key is $(p,q,v_p,v_q)$.

\subsubsection{Encryption and Decryption}
To encrypt a message $m \in \mathbb{Z}_u$, the encryption function is defined as:
\begin{align*}
  E(m) = g^m \cdot h^r \mod{n},
\end{align*}
where $r$ is a random integer in $\mathbb{Z}_n$ which is longer than $2t$ bits.

To decrypt a ciphertext $c \in \mathbb{Z}_n^\ast$, decryption is achieved by first computing $c^{v_pv_q}$.
\begin{align*}
	c^{v_pv_q} \bmod n
	&= (g^m \cdot h^r)^{v_pv_q} \bmod n\\
	&= (g^{v_pv_q})^m \bmod n.
\end{align*}
Since $(g^{v_pv_q})^m$ has order $u$, there is a one-to-one correspondence between plaintexts in $\mathbb{Z}_u$ and  $(g^{v_pv_q})^m$. A lookup table can thus be generated privately to successfully recover $m$.

\subsubsection{Homomorphic Properties of the DGK Cryptosystem}
The DGK cryptosystem supports additive homomorphism as well as the multiplication of a plaintext scalar to an encrypted message. These operations are defined as follows.
For all $m_1,m_2 \in \mathbb{Z}_u$ and $k\in \mathbb{N}$, the following homomorphic properties hold.
\begin{align*}
    D(E(m_1)g^k)=(m_1+k)\bmod u & \text{ (add a plaintext constant)}\\
    D(E(m_1)E(m_2))=(m_1+m_2)\bmod u & \text{ (ciphertext addition)}\\
    D(E(m_1)^k)= km_1\bmod u & \text{ (multiply a plaintext constant)}
\end{align*}

As the multiplicative homomorphism was not presented in the original paper \cite{pieprzyk_efficient_2007, cryptoeprint:2008:321}, we provide a short proof here.
\begin{proof}
  Let $m \in \mathbb{Z}_u$ and $k\in \mathbb{N}$.
  We consider $E(m)^k = (g^m \cdot h^r \bmod{n})^k\bmod n$.
  \begin{align*}
    (g^m \cdot h^r \bmod{n})^k \bmod n
    &= (g^m \cdot h^r)^k \bmod{n}\\
    &= (g^m)^k \cdot (h^r)^k \bmod{n}\\
    &= g^{km} \cdot (h^{kr}) \bmod{n}
  \end{align*}
  Since $r$ is a random integer, $kr$ is also a random integer, Therefore, $g^{km} \cdot (h^{kr}) \bmod{n} = E(m)^k$ is a valid encryption of the message $km$.
\end{proof}

\subsection{Floating-Point Arithmetic using PHE}
\label{sec:fp_arithmetic}
We have discussed the Paillier and DGK cryptosystems, which encrypt and allow similar operations on encrypted integers. We will now show how these PHE schemes may be extended to a system which allows for secure floating-point computation. In this section, we let $\oplus$ and $\otimes$ represent the homomorphic operations corresponding to the secure addition and multiplication of integers in a PHE scheme, respectively. 

\subsubsection{Extension to Floating-Point Numbers}
\label{sec:fp_number_extension}
We can use the following protocol, described in~\cite{ziad_cryptoimg:_2016}, in order to extend Paillier and DGK to floating-point numbers.

We represent a floating-point (FP) number as a pair of two integers $(m,e)$ representing the mantissa and exponent of the FP number with respect to a base $b$. The mantissa $m$ is encrypted, while the exponent $e$ is unencrypted.
Let $a,b,c$ be FP numbers represented by the pairs $(m_a,e_a),(m_b,e_b),(m_c,e_c)$ respectively. We define the corresponding FP number operations as follows:
\begin{description}
  \item[Addition.]
    To compute $E\left(c\right)=E\left(a+b\right)$ we compute
	\begin{align*}
		E\left(m_c\right) &= 
		\begin{cases}
			E\left(m_a\right) \oplus \left(b^{e_b-e_a} \otimes E\left(m_b\right)\right) & \text{if } e_a \leq e_b \\
			E\left(m_b\right) \oplus \left(b^{e_a-e_b} \otimes E\left(m_a\right)\right) & \text{if } e_a > e_b
		\end{cases}, \\
		e_c &= 
		\begin{cases}
			e_a & \text{if } e_a \leq e_b \\
			e_b & \text{if } e_a > e_b
		\end{cases}.
	\end{align*}
  \item[Scalar multiplication.]
    To compute $E\left(c\right) = E\left(ab\right)$, where $a$ and $E\left(b\right)$ are known (i.e., $m_a$ is not encrypted), we compute
    \begin{align*}
      E\left(m_c\right) &= m_a \otimes E\left(m_b\right),\\
      e_c &= e_a + e_b.
    \end{align*}
\end{description}

\subsubsection{Secure Division}
We can use the following two-party scheme defined in~\cite{boukoros_lightweight_2017} to perform privacy-preserving division.

Suppose Bob has $E\left(x\right)$ and $E\left(y\right)$ and wants to obtain $E\left(x/y\right)$ without exposing the value of either variable.
\begin{itemize}
	\item Bob first selects a random non-zero number $r$ and computes $E\left(rx\right)$ and $E\left(ry\right)$. Bob can do this since $r$ is a plaintext constant.
	\item Bob sends $E\left(rx\right)$ and $E\left(ry\right)$ to Alice, who decrypts both values and computes $x/y$ in the plaintext domain.
	\item Alice encrypts and sends $E\left(x/y\right)$ to Bob.
\end{itemize}

\subsubsection{Secure Exponentiation}
\label{ssec:exponentiationprotocol}
 We can use the following two-party scheme, adapted from the protocol to calculate Euclidean distances used in \cite{hutchison_privacy-preserving_2009}, in order to perform secure exponentiation.

Suppose Alice encrypts an integer $x$ and sends it so Bob has $E\left(x\right)$, and wants to obtain $E\left(x^2\right)$ without exposing the value of $x$.
\begin{itemize}
	\item Bob first selects a random integer $r$ and computes $E\left(x+r\right)$. Bob can do this since $r$ is a plaintext constant.
	\item Bob sends $E\left(x+r\right)$ to Alice, who decrypts the ciphertext to obtain $x+r$.
	\item Alice squares $x+r$ and encrypts the result. She sends $E\left(\left(x+r\right)^2\right)$ to Bob.
	\item Bob computes $E\left(-2rx + r^2\right)$. He then computes
	\begin{align*}
		E\left(\left(x+r\right)^2+ \left(-2rx + r^2\right)\right) = E\left(x^2\right).
	\end{align*}
\end{itemize}

\subsubsection{Secure Multiplication}
We can use the secure squaring protocol to arrive at a secure multiplication protocol, which then allows for the evaluation of polynomials.
Suppose Alice encrypts integers $x$ and $y$, and Bob has $E\left(x\right), E\left(y\right)$ and wants to obtain $E\left(xy\right)$.
\begin{itemize}
	\item Bob acquires $E\left(x^2\right)$ and $E\left(y^2\right)$ using the secure squaring protocol.
	\item Bob sends $E\left(x+y\right)$ to Alice, who decrypts the ciphertext to obtain $x+y$.
	\item Alice sends $E\left(\left(x+y\right)^2\right)$ to Bob.
	\item Bob then computes
	\begin{align*}
		E\left(\frac{1}{2}\left(\left(x+y\right)^2\ - x^2 - y^2\right)\right) 
		&= E\left(xy\right).
	\end{align*}
\end{itemize}

By applying these extensions to the Paillier and DGK cryptosystems, privacy-preserving floating-point arithmetic can be acheived with PHE schemes. 
