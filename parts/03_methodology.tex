\chapter{METHODOLOGY}

After introducing your topic of choice, discussing and relating previous work with your own, and presenting the underlying concepts that your study will be working with, this section will enable you to go into fine detail into how you will go about your study. Essentially, whatever data you need to gather, as well as how you will intend to gather them, should be presented here.
In order to check whether or not your methodology is sound, two main questions should be answered:
\begin{enumerate}
	\item Is your methodology replicable?
	\item Is your methodology realistic and time-bound?
\end{enumerate}

\section{Methodology as Replicable}
A replicable methodology basically means that anyone who reads your methodology and intends to recreate your study to the letter must be able to obtain a similar, if not, exactly the same set of results. It is important, therefore, that you be as specific as you can when describing your methods, such as properly delineating your study’s independent, dependent, and control variables. Much like in the literature review and framework, it is good practice to organize your methodology into subsections for easier readability. Of course, apart from generating data given these variables, included in making the methodology replicable is providing the users an effective and appropriate means to collect data for analysis later on. This assumes, of course, that the data you intend to collect is actually measurable, whether it be quantitative (numerical) or qualitative (descriptive).

\section{Methodology as Realistic and Time-Bound}
On the other hand, a realistic and time-bound methodology takes into consideration the context of the researcher. Although a high-level of competency is expected from a graduating CS major, one must also ensure that the proposed study’s level of difficulty is aligned with what limited resources is available, especially time. In fact, given that the trend is that you will undergo actual implementation only after being able to defend your proposal during the first semester, the study should be accomplishable at the most within only a semester. It is therefore imperative in the methodology, especially in its initial presentation during the defense, that the timetable for the study is thoroughly laid out, with workable time frames and specific dates for deliverables.

\section{Summary and Additional Guide Questions}
The methodology, in summary, is your detailed explanation of how you intend to go about implementing your study.