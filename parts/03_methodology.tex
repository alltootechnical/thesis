\chapter{METHODOLOGY}

The study consists of two parts: first, the implementation of the three homomorphic encryption schemes and image processing operations under each scheme as an OpenCV library. The second part of the study consists of assessing the robustness and security of image operations under each encryption scheme using benchmarks in \cite{ahmed_benchmark_2016}.

\section{Implementation of the OpenCV library}

The current version of the OpenCV library (3.4.1) will be forked from the open-source GitHub repository at https://github.com/opencv \cite{opencv_library}.

Three homomorphic encryption schemes will be implemented, those presented by Ziad, et al. \cite{ziad_cryptoimg:_2016}, Li, et al. \cite{li_elliptic_2012} and Smart and Vercauteren \cite{hutchison_fully_2010}. The encryption and decryption methods will be implemented in C++.

Aside from implementing the homomorphic encryption and decryption algorithms, we will also implement library functions for the following image processing functions, as they are defined in \cite{gonzalez_digital_2008}. We let $R(x,y)$ denote the intensity at coordinate $(x,y)$ in the source image, and $S(x,y)$ denote the intensity at coordinate $(x,y)$ in the resulting image. We further suppose that the intensity values of pixels are in the range $[0, L-1]$.
\begin{description}
	\item[Intensity transformations.] Transformations on the intensities of each of the pixels on an image. The following definitions hold for all $x,y$.
	\begin{enumerate}
		\item Image negation: $S(x,y) = L - 1 - R(x,y)$.
		\item Log transformation: $S(x,y) = c\log{(1 + R(x,y))}$, $c \geq 0$.
		\item Power-law transformation: $S(x,y) = c[R(x,y)]^\gamma$, $c > 0, \gamma > 0$.
	\end{enumerate}
	\item[Spatial filters.] Filters implemented by performing a convolution between an $M\times N$ source image and an $m\times n$ filter matrix. Let $W$ be a filter matrix. Then the corresponding spatial filter is given by
	\begin{equation}
		S(x,y) = \sum_{s=1}^m{\sum_{t=1}^n{W(s,t)R(x+s,y+t)}}
	\end{equation}
	\item [Histogram processing / equalization]

\end{description}

\section{Assessment of each homomorphic encryption scheme}

We will first obtain standard test images from \cite{gonzalez_image_nodate}. Both greyscale and color images will be used.

For each image, we will consider each image operation listed above and generate two images: a plaintext domain transformation (PDT) and an encrypted domain transformation (EDT). The PDT will be generated by running the image operation on the

The benchmarks to be used in the study, adopted from \cite{ahmed_benchmark_2016, ahmad_efficiency_2012, wu_npcr_2011} are listed below. We let $X_i$ denote from
\begin{enumerate}
	\item Tests for preservation of image quality after encryption and decryption.
	\begin{enumerate}
		\item Mean Squared Error (MSE)

		\item Peak Signal to Noise Ratio (PSNR)
		\item Structural Similarity Index (SSIM)
		\item Noise tolerance
	\end{enumerate}
	\item Tests for cryptographic security
	\begin{enumerate}
		\item Information entropy analysis
		\item Correlation coefficient analysis
		\item Differential analysis (number of pixel change rate (NPCR), universeral average change intensity (UACI))
	\end{enumerate}
\end{enumerate}
