\chapter{METHODOLOGY}

The study consists of two parts: first, the implementation of the three homomorphic encryption schemes and image processing operations under each scheme as an OpenCV library. The second part of the study consists of assessing the robustness and security of image operations under each encryption scheme using benchmarks in \cite{ahmed_benchmark_2016}.

\section{Implementation of the OpenCV library}

The current version of the OpenCV library (3.4.1) will be forked from the open-source GitHub repository at https://github.com/opencv \cite{opencv_library}.

Three homomorphic encryption schemes will be implemented, those presented by Ziad, et al. \cite{ziad_cryptoimg:_2016}, Li, et al. \cite{li_elliptic_2012} and Smart and Vercauteren \cite{hutchison_fully_2010}. The encryption and decryption methods will be implemented in C++.

Aside from implementing the homomorphic encryption and decryption algorithms, we will also implement library functions for the following image processing functions, as they are defined in \cite{gonzalez_digital_2008}. We will also take note of whether or not a candidate operation is impossible to perform under a given cryptosystem.  We let $R(x,y)$ denote the intensity at coordinate $(x,y)$ in the source image, and $S(x,y)$ denote the intensity at coordinate $(x,y)$ in the resulting image. We further suppose that the intensity values of pixels are in the range $[0, L-1]$.
\begin{description}
	\item[Intensity transformations.] Transformations on the intensities of each of the pixels on an image. The following definitions hold for all $x,y$.
	\begin{enumerate}
		\item Image negation: $S(x,y) = L - 1 - R(x,y)$.
		\item Log transformation: $S(x,y) = c\log{(1 + R(x,y))}$, $c \geq 0$.
		\item Power-law transformation: $S(x,y) = c[R(x,y)]^\gamma$, $c > 0, \gamma > 0$.
	\end{enumerate}
	\item[Spatial filters.] Filters implemented by performing a convolution between an $M\times N$ source image and an $m\times n$ filter matrix. Let $W$ be a filter matrix. Then the corresponding spatial filter is given by
	\begin{align}
		S(x,y) = \sum_{s=1}^m{\sum_{t=1}^n{W(s,t)R(x+s,y+t)}}.
	\end{align}
	\item [Histogram processing / equalization]

\end{description}

\section{Assessment of each homomorphic encryption scheme}

We will first obtain standard test images from \cite{gonzalez_image_nodate}. Both greyscale and color images will be used.

For each plaintext image (PT), we will consider each image operation listed above and generate three images: a plaintext domain transformation (PDT), a ciphertext image (CT), and an encrypted domain transformation (EDT). The cipherThe PDT will be generated by running the image operation on the original image. The CT will be generated by encrypting the image, then applying the operation, and the EDT will be generated by decrypting the CT. The four images (PT, PDT, CT, CDT) will then be compared using various benchmarks to evaluate the quality and security of each homomorphic encryption scheme.

The benchmarks to be used in the study, adopted from \cite{ahmed_benchmark_2016, ahmad_efficiency_2012, wu_npcr_2011} are listed below. We let $X_i$ denote a value in an image $X$, where $1 \leq i \leq N$.

We first perform three tests to ascertain the preservation of image quality after encryption and decryption: MSE, PSNR, and SSIM.
\begin{description}
	\item [Mean Squared Error (MSE).] The MSE is defined \cite{ahmed_benchmark_2016} as
	\begin{align}
		MSE = \frac{1}{N}\sum_{i=1}^{N}{(CDT_i - PDT_i)^2}.
	\end{align}
	The MSE provides a measure of how much data is recovered if an image operation is applied on the encrypted image, which is then decrypted. Lower values of MSE indicate higher preservation of image quality \cite{ahmed_benchmark_2016, ahmad_efficiency_2012}.
	\item [Peak Signal to Noise Ratio (PSNR).]
	According to Ahmed, et al. \cite{ahmed_benchmark_2016}, PSNR is ``an estimator for human visual perception of reconstruction quality'' which is based on  It has been used to ascertain image quality in various studies and is a known meetric for image and video quality \cite{upmanyu_efficient_2009, ahmed_benchmark_2016, akramullah_video_2014}. Although it may produce results which do not correlate with human visual perception \cite{huynh-thu_accuracy_2012, ahmed_benchmark_2016}, it is a valid indicator of image quality when media containing the same visual content is compared \cite{huynh-thu_accuracy_2012}.
	PSNR is defined by
	\begin{align}
		PSNR = 10\log_{10}{\left( \frac{L^2}{MSE} \right)}
	\end{align}
	where $L$ is the maximum pixel intensity value of an image.
	Despite the known limitations of PSNR, since we are going to compare the effect of each encryption scheme on recovered image quality, given a fixed library of images, it is a valid measure of image quality for the study.
	\item [Structural Similarity Index (SSIM)]
	The SSIM for two random variables $X$ and $Y$ is defined \cite{ahmed_benchmark_2016, akramullah_video_2014} as
	\begin{align}
		SSIM(X,Y) = \frac{(2\mu_X\mu_Y+c_1)(2\sigma_{XY}+c_2)}{(\mu_X^2+\mu_Y^2+c_1)(\mu_X^2+\mu_Y^2+c_2)}
	\end{align}
	where
	\begin{itemize}
		\item $\mu_X, \mu_Y$ are the averages of $X$ and $Y$, respectively;
		\item $\sigma_X, \sigma_Y$ are the variances of $X$ and $Y$, respectively;
		\item $\sigma_{XY}$ is the covariance of $X$ and $Y$;
		\item $c_1 = (k_1L)^2, c_2 = (k_2L)^2$ are two variables used to stabilize the measure when $\mu_X^2+\mu_Y^2$ is close to zero \cite{akramullah_video_2014};
		\item $L$ is the the maximum pixel intensity value of an image;
		\item $k_1 = 0.01, k_2 = 0.03$ by default, given in \cite{ahmed_benchmark_2016}.
	\end{itemize}
	The SSIM is applied to the luminance value of two images to gauge structural similarity between neighboring pixels.
	For the study, we will compute $SSIM(PDT, CDT)$ for every image operation, under each homomorphic cryptosystem.
\end{description}

After performing tests to evaluate image quality, we then perform tests for cryptographic security, as given in \cite{ahmed_benchmark_2016}: entropy analysis, correlation coefficient analysis, NPCR and UACI.
\begin{description}
	\item [Information entropy analysis]
	\item [Correlation coefficient analysis]
	\item [Number of Pixel Change Rate (NPCR)]
	\item [Universeral Average Change Intensity (UACI)]
\end{description}

\section{Summary and Timetable}
