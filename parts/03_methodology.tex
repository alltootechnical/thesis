\chapter{METHODOLOGY}

In this study we apply four homomorphic encryption schemes~\cite{ziad_cryptoimg:_2016, pieprzyk_efficient_2007, dasgupta_design_2016, garay_algorithms_2014} for use in image processing and facial recognition applications.
While the Paillier and DGK cryptosystems have been implemented for use in image processing in~\cite{ziad_cryptoimg:_2016, hutchison_privacy-preserving_2009}, the other cryptosystems have to be adapted for this study.
Two experiments were conducted in this study:
\begin{enumerate}
	\item Assessment of non-linear image intensity transformations under each of the four cryptosystems.
	\item Assessment of efficiency of each of the four cryptosystems when applied for use in privacy-preserving facial recognition and detection using eigenfaces.
\end{enumerate}
% redundant sentence, this is explained later:
% The results of the image intensity transformations were tested for image quality using benchmarks in~\cite{ahmed_benchmark_2016}.

\section{Cryptosystem Implementation}
% The current version of the OpenCV library (3.4.2) will be forked from the open-source GitHub repository at \url{https://github.com/opencv/opencv})~\cite{bradski_opencv_2000}.

Four homomorphic encryption schemes were investigated. For cryptosystems with readily available open-source implementations, such implementations were used. For the remaining cryptosystems, the encryption and decryption methods were implemented in Python.
\begin{itemize}
	\item The Paillier cryptosystem is incorporated using the \texttt{python-paillier} library at \url{https://github.com/n1analytics/python-paillier} developed by Data61 - CSIRO.
	\item The DGK cryptosystem is ported from the C++ implementation at \url{https://github.com/encryptogroup/ENCRYPTO_utils}, developed by Daniel Demmler from EC-SPRIDE.
	\item The DGHV cryptosystem is ported from the C++ implementation at 
	\url{https://github.com/deevashwer/Fully-Homomorphic-DGHV-and-Variants}.
	\item The BGV cryptosystem is implemented using the Pyfhel library (\url{https://github.com/ibarrond/Pyfhel}) using a HElib backend from (\url{https://github.com/shaih/HElib}).
\end{itemize}

For convenience, a wrapper library was created that integrates the four cryptosystems into a unified interface.

The Paillier and DGK cryptosystems were adapted for floating-point computations using the protocols described in section \ref{sec:fp_arithmetic}.

%We will also take note of whether or not a candidate operation is impossible to perform under a given cryptosystem.  We let $R(x,y)$ denote the intensity at coordinate $(x,y)$ in the source image, and $S(x,y)$ denote the intensity at coordinate $(x,y)$ in the resulting image. We further suppose that the intensity values of pixels are in the range $[0, L-1]$.
%\begin{description}
%	\item[Intensity transformations.] Transformations on the intensities of each of the pixels on an image. The following definitions hold for all $x,y$.
%	\begin{enumerate}
%		\item Image negation: $S(x,y) = L - 1 - R(x,y)$.
%		\item Log transformation: $S(x,y) = c\log{(1 + R(x,y))}$, $c \geq 0$.
%		\item Power-law transformation: $S(x,y) = c[R(x,y)]^\gamma$, $c > 0, \gamma > 0$.
%	\end{enumerate}
%	\item[Spatial filters.] Filters implemented by performing a convolution between an $M\times N$ source image and an $m\times n$ kernel. Let $k$ be a kernel. Then the corresponding spatial filter is given by
%	\begin{align}
%		S(x,y) = \sum_{s=1}^m{\sum_{t=1}^n{k(s,t)R(x+s,y+t)}}.
%	\end{align}
%	Morphological operations such as erosion and dilation can be achieved using convolution as well.
%\end{description}

\section{Non-Linear Intensity Transformations}
Aside from implementing the homomorphic encryption and decryption algorithms, we will also implement library functions for image negation, logarithm transformation and power-law transformation, as they are defined in~\cite{gonzalez_digital_2008}.
We now describe how we implement the logarithm and power-law image intensity transformations. In the case of partially homomorphic cryptosystems (Pailler, DGK), the following considerations will be made regarding required homomorphic operations not supported by the cryptosystems:
\begin{enumerate}
	\item If exponentiation is required, the secure exponentiation protocol in section \ref{ssec:exponentiationprotocol} will be used.
	\item If multiplication of ciphertexts $E(m_1), E(m_2)$ is required, the following protocol will be used:
	\begin{itemize}
		\item Bob calculates $E(m_1+m_2)$ and using the exponentiation protocol, acquires $E(m_1^2),E(m_2^2),E((m_1+m_2)^2)$.
		\item Bob can recover the product of the ciphertexts by computing
		\begin{align*}
			E((m_1+m_2)^2) - E(m_1^2) - E(m_2^2) = E(2m_1m_2),
		\end{align*}
		then carrying out the required constant multiplication to obtain $E(2m_1m_2)$.
	\end{itemize}
\end{enumerate}
\subsection{Logarithm Transformation}
The logarithm transformation of a pixel intensity value $x$ is defined as
\begin{align}
	T\left(x\right) = c \log\left(1 + x\right)
\end{align}
where $c$ is a constant.

In order to perform this transformation under a homomorphic cryptosystem, we must provide an approximation for $\log\left(1 + x\right)$ in terms of addition and multiplication operations (or their inverses). We have derived a closed form approximation in Appendix \ref{sec:logapproximation}.
\begin{align}
	\label{eq:scaledquadraturech3}
  \begin{split}
    &\log(1+x) \\
    &=\frac{137x^5 + 33185x^4 + 931370x^3 - 13403630x^2 - 289469315x - 713567363}
    {30(x^5 + 505x^4 + 42010x^3 + 923010x^2 + 5722005x + 8040501)} + \log{20}
  \end{split}
\end{align}

\subsection{Power-Law Transformation}
The power-law transformation of a pixel intensity value $x$ is defined as
\begin{equation}
    T\left(x\right) = cx^{\gamma}
\end{equation}
where $c>0$ and $\gamma > 0$.

Similar to the logarithm transformation, we must provide an approximation for $x^\gamma$.
We have derived an infinite series in Appendix \ref{sec:logapproximation}.
\begin{align*}
	x^\gamma &= \sum_{n=0}^{\infty}{\frac{(\gamma\log{x})^n}{n!}}
\end{align*}
Partial sums of the above infinite series can be calculated based on the closed form approximation for the logarithm in Equation \ref{eq:scaledquadraturech3}.


\section{Assessment of Homomorphic Encryption Schemes}

A laptop computer with a 2.5 GHz Intel Core i7 quad-core processor and 8 GB of RAM is used to perform the computations, and processing time was tracked using built-in timing functions. The processing time for all cases were recorded. All of the tests are done using Python 3.7.1 running on Linux.

\subsection{Image Quality of Intensity Transformations}
In this section, we discuss the image quality metrics we used for assessing the performance of non-linear intensity transformations.

We first obtained test images from the \texttt{faces94} dataset (\url{https://cswww.essex.ac.uk/mv/allfaces/faces94.html}) maintained by the University of Essex. However, only ten distinct facial images from the dataset were selected for the purposes of this test. Since only grayscale images are used in the study, the images were converted to grayscale and then downsampled to $40 \times 36$ pixels.

For each cryptosystem, three image operations are tested:
\begin{enumerate}
	\item Image negation
	\item Logarithm transformation, with $c = 30$
	\item Power-law transformation, with $c = 1$ and $\gamma = 0.4$
\end{enumerate}

For each plaintext image (PT), we considered each image operation listed above and generate three images: a plaintext domain transformation (PDT), a ciphertext image (CT), and an encrypted domain transformation (EDT). The PDT was generated by running the image operation on the original image. The CT was generated by encrypting the image, then applying the operation, and the EDT was generated by decrypting the CT. The four images (PT, PDT, CT, EDT) were then compared using various benchmarks to evaluate the quality and security of each homomorphic encryption scheme.

The benchmarks to be used in the study, adopted from~\cite{ahmed_benchmark_2016, ahmad_efficiency_2012, wu_npcr_2011} are listed below. We let $X_i$ denote a value in an image $X$, where $1 \leq i \leq N$.
We first perform three tests to ascertain the preservation of image quality after encryption and decryption: MSE, PSNR, and SSIM.
\begin{description}
	\item [Mean Squared Error (MSE).] The MSE is defined in~\cite{ahmed_benchmark_2016} as
	\begin{align}
        \mathrm{MSE} = \frac{1}{N}\sum_{i=1}^{N}{(\mathrm{CDT}_i - \mathrm{PDT}_i)^2}.
	\end{align}
	The MSE provides a measure of how much data is recovered if an image operation is applied on the encrypted image, which is then decrypted. Lower values of MSE indicate higher preservation of image quality~\cite{ahmed_benchmark_2016, ahmad_efficiency_2012}.
	\item [Peak Signal to Noise Ratio (PSNR).]
	According to Ahmed, et al.~\cite{ahmed_benchmark_2016}, PSNR is ``an estimator for human visual perception of reconstruction quality.'' It has been used to ascertain image quality in various studies and is a known metric for image and video quality~\cite{upmanyu_efficient_2009, jain_image_2016, akramullah_video_2014}. Although it may produce results which do not correlate with human visual perception~\cite{huynh-thu_accuracy_2012, ahmed_benchmark_2016}, it is a valid indicator of image quality when media containing the same visual content is compared~\cite{huynh-thu_accuracy_2012}.
	PSNR is defined by
	\begin{align}
        \mathrm{PSNR} = 10\log_{10}{\left( \frac{L^2}{\mathrm{MSE}} \right)}
	\end{align}
	where $L$ is the maximum pixel intensity value of an image.
	Despite the known limitations of PSNR, since we are going to compare the effect of each encryption scheme on recovered image quality, given a fixed library of images, it is a valid measure of image quality for the study. A higher PSNR indicates higher image quality preservation.
	\item [Structural Similarity Index (SSIM).]
	The SSIM for two random variables $X$ and $Y$ is defined in~\cite{ahmed_benchmark_2016, akramullah_video_2014} as
	\begin{align}
        \mathrm{SSIM}(X,Y) = \frac{(2\mu_X\mu_Y+c_1)(2\sigma_{XY}+c_2)}{(\mu_X^2+\mu_Y^2+c_1)(\mu_X^2+\mu_Y^2+c_2)}
	\end{align}
	where
	\begin{itemize}
		\item $\mu_X, \mu_Y$ are the averages of $X$ and $Y$, respectively;
		\item $\sigma_X, \sigma_Y$ are the variances of $X$ and $Y$, respectively;
		\item $\sigma_{XY}$ is the covariance of $X$ and $Y$;
		\item $c_1 = (k_1L)^2, c_2 = (k_2L)^2$ are two variables used to stabilize the measure when $\mu_X^2+\mu_Y^2$ is close to zero~\cite{akramullah_video_2014};
		\item $L$ is the the maximum pixel intensity value of an image;
		\item $k_1 = 0.01, k_2 = 0.03$ by default, given in~\cite{ahmed_benchmark_2016}.
	\end{itemize}
	The SSIM is applied to the luminance value of two images to gauge structural similarity between neighboring pixels.
	For the study, we computed $\mathrm{SSIM}(\mathrm{PDT}, \mathrm{CDT})$ for every image operation, under each homomorphic cryptosystem. Higher values of SSIM indicate higher structural similarity, and an SSIM of $1$ indicates that the two images are identical~\cite{ahmed_benchmark_2016}.
\end{description}

Computing for MSE, PSNR, and SSIM was done using the corresponding built-in functions from the scikit-image Python library \cite{scikit-image}.

\subsection{Facial Recognition Tests}
%% TODO: Brian, can you fill this part up? Yes. OK Thanks
The method for secure facial recognition using eigenfaces by Erkin, et al. \cite{hutchison_privacy-preserving_2009-2} was implemented for this study.
%Training data will be secured from The Database of Faces (\url{https://www.cl.cam.ac.uk/research/dtg/attarchive/facedatabase.html}) maintained by AT&T Laboratories Cambridge.
Training and test data were secured from the \texttt{faces94} dataset (\url{https://cswww.essex.ac.uk/mv/allfaces/faces94.html}) maintained by the University of Essex. 

% Haar cascades will be implemented using homomorphic operations for facial detection.

% Each cryptosystem will be tested on accuracy (compared to unencrypted facial detection and recognition) and processing speed.

The facial recognition tests were done using ten-fold cross-validation for each cryptosystem, where the number of principal components used by the algorithm is fixed at $K=5$. 

Prior to splitting the dataset, the images within each set are shuffled. Since there are twenty (20) images per set, two images are taken at a time for each round of cross-validation. The first round would take the first two images from each set as part of the test set, and the rest of the images will be part of the training set. Then, the succeeding round would take the next two images from each set as part of the test set, and so on until ten rounds have been done. After every round of cross-validation, the accuracy score, confusion matrix, and total processing time were obtained. 

% \section{Summary}
% In summary, the study consists of:
% \begin{itemize}
% 	\item Creating a wrapper library that acts as a unified interface which implements the Paillier, DGK, DGHV, and BGV homomorphic cryptosystems.
% 	\item An implementation of image processing operations (intensity transformations, facial recognition) under the above cryptosystems.
% 	\item Conducting tests for image quality, processing time, using standard test images from~\cite{gonzalez_image_nodate}.
% \end{itemize}

%% insert flowchart/diagrams here
% sure
