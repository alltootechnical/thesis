\chapter{REVIEW OF RELATED LITERATURE}

We will now provide an overview of the \textit{CryptoImg} project, which our study is primarily based on. Then, we define the image processing operations we are interested in implementing in this study and
After establishing our desired image processing operations, introduce concepts in cryptography and define the four homomorphic cryptosystems we will be testing in the study.

\section{CryptoImg}
A paper published in 2009 by Ziad, et al. attempts to implement privacy-preserving image processing using \textit{CryptoImg}, a library for the Open Source Computer Vision Library (OpenCV)~\cite{bradski_opencv_2000} which implements various homomorphic encryption and image processing routines using the Paillier homomorphic cryptosystem~\cite{ziad_cryptoimg:_2016}. The CryptoImg library assumed a client-server model wherein a client requests image processing operations from a server. A client first encrypts a digital image and sends the image securely to a server, which then operates on the encrypted image without revealing its contents. Then the resulting image is sent back to the client, which decrypts the image to recover the desired output.
\begin{figure*}[!ht]
    \centering
    \includegraphics[width=\textwidth,\keepaspectratio]{figures/ClientServerModel.png}
    \caption{Client-server architecture used by \textit{CryptoImg} \cite{ziad_cryptoimg:_2016}}
    \label{fig:clientserver}
\end{figure*}
The \textit{CryptoImg} library implemented the following:
\begin{enumerate}
	\item Extending the Paillier homomorphic cryptosystem which operate on integer plaintexts so that they can operate on real number plaintexts;
	\item Developing protocols for and implementing the following image processing operations
	\begin{enumerate}
		\item Image negation and brightness adjustment
		\item Spatial filters (for noise reduction, edge detection and sharpening)
		\item Morphological operations
		\item Histogram equalization
	\end{enumerate}
\end{enumerate}
For image negation and spatial filters, the protocols specified by \textit{CryptoImg} allow all image processing operations to be performed on the server. However, due to limitations in the Paillier cryptosystem, the protocols presented for morphological operations and histogram equalization require both the client and the server to perform image processing calculations, although the server performs a significant portion of the processing.

Ziad, et al. also showed experimental results establishing the slow performance of image operations under a homomorphic cryptosystem. For instance, while sharpening and applying a Sobel filter each take less than a second when applied to a $512\times 512$ plaintext image, when applied to an encrypted image, sharpening required at least 238.257 seconds, and applying the Sobel filter required at least 147.567 seconds \cite{ziad_cryptoimg:_2016}.

%%limitations = not considering transcendental function evaluation
%%extension to fully homomorphic cryptosystems
%%combined applications in face detection and recognition

We now discuss several limitations in the \textit{CryptoImg} study which we focus on for our research. First, the \textit{CryptoImg} library was limited in the image intensity transformations it implemented. We propose additional protocols to support more computationally intensive intensity transformations.
Second, the \textit{CryptoImg} library only considered the Paillier cryptosystem. We consider testing the performance of other homomorphic cryptosystems, which differ in their processing time and supported operations.

\section{Common Image Operations}
In this study, we wish to see how the primitive operations of addition and multiplication supported by homomorphic cryptosystems can be applied to image processing operations. These operations are performed on digital images. WE can represent a digital image $A$ as an $M \times N$ matrix of pixel intensity values, each value in the range $\left[0, L-1\right]$, for some positive integer $L$. There are two types of basic operations in image processing, intensity transformations, which map an intensity value to another, spatial filters which assist in operations such as edge detection and image blurring.

\subsection{Intensity Transformation}
To define an intensity transformation on an image $A$, we define a function $T$ which maps a pixel value $r$ to a new value $r^\prime$, which we can write as $r^\prime = T\left(r\right)$. This function is then applied to every pixel in $A$. Examples of intensity transformations are image negation, log transformation, and power-law transformation.

Image negation is an example of an intensity transformation, where the resulting image would be similar to a photographic negative~\cite{gonzalez_digital_2008}. An image negation operation is defined as:
\begin{equation}
    T\left(r\right) = L-1-r
\end{equation}

The log transformation is used to enhance dark pixels or increase the dark details of an image by mapping low intensity values to a wider range of values~\cite{gonzalez_digital_2008}. This has the general form
\begin{equation}
    T\left(r\right) = c \log\left(1 + r\right)
\end{equation}
where $c$ is a constant and $r \ge 0$.

The power-law transformation is a family of transformations that have the form
\begin{equation}
    T\left(r\right) = c r^{\gamma}
\end{equation}
where $c>0$ and $\gamma > 0$. This is especially useful since many image capture and output devices such as cameras, printers and displays follow a similar power law to relate physical percieved light intensity and digital pixel intensity values. A power-law transformation defined by the above equation can calibrate the operation on these devices in a process called \textit{gamma correction}. This ensures reproducibility and accuracy of images being displayed~\cite{gonzalez_digital_2008}.

\subsection{Edge Detection and Spatial Filtering}
Edge detection is used to find and determine the boundaries in an image, commonly used in applications such as image segmentation and feature extraction. This works by detecting so-called \textit{edges}, areas that have abrupt changes in intensity.

Edge detection is usually done by using gradient operators that detect such abrupt changes. These operators are commonly known as \textit{spatial filter}, which are usually of $3 \times 3$ size. A common example of spatial filters is the Sobel operator, with two matrices (also called as kernels) $g_x$ and $g_y$ representing the horizontal and vertical components respectively.
\begin{equation}
    g_x =
    \begin{bmatrix}
        -1 & 0 & 1 \\
        -2 & 0 & 2 \\
        -1 & 0 & 1
    \end{bmatrix}
    \qquad\text{and}\qquad
    g_y =
    \begin{bmatrix}
        1 & 2 & 1 \\
        0 & 0 & 0 \\
        -1 & -2 & -1
    \end{bmatrix}
\end{equation}
To get the resulting image $I^\prime$, a convolution is performed between the original image $I$ of size $M \times N$ and the kernel $k$ of size $m \times n$. Now suppose that the pixel value of an image at point $\left(i,j\right)$ is $r_{i,j}$. Then, a transformation using spatial filters can be described as follows:
\begin{align}
    T\left(r_{i,j}\right) &= \left[k * I\right]\left(\left\lfloor\frac{m}{2}\right\rfloor, \left\lfloor\frac{n}{2}\right\rfloor \right) \\
                         &= \sum_{u=1}^{m} \sum_{v=1}^{n} \left[k_{i,j} r_{i+u, j+v} \right]
\end{align}

Spatial filters are not only used for edge detection, but there are also filters that do image smoothing (such as Gaussian blur and box blur, $b_g$ and $b$ respectively in Equation~\ref{eqn:smooth-filters}) and image sharpening, to name a few \cite{gonzalez_digital_2008}.
\begin{equation}
    \label{eqn:smooth-filters}
    b_g = \frac{1}{16}
    \begin{bmatrix}
        1 & 2 & 1 \\
        2 & 4 & 2 \\
        1 & 2 & 1
    \end{bmatrix}
    \qquad
    b = \frac{1}{9}
    \begin{bmatrix}
        1 & 1 & 1 \\
        1 & 1 & 1 \\
        1 & 1 & 1
    \end{bmatrix}
\end{equation}
% Should I put a table of other kernels too?

%facial recognition - theory
\section{Facial Detection and Recognition}
%% TODO: Brian dump ur stuff here

\section{Homomorphic Cryptosystems}

In cryptography, a cryptosystem consists of an encryption function $\mathcal{E}$ and a decryption function $\mathcal{D}$, along with the plaintext space $\mathcal{P}$, ciphertext space $\mathcal{C}$ and the key space $\mathcal{K}$~\cite{bauer_cryptosystem_2005}. A \textit{plaintext} is text that can be commonly understood, while a \textit{ciphertext} results from encrypting the plaintext using an encryption key. The \textit{plaintext space} is the set of all possible plaintexts, the \textit{ciphertext space} is the set of all possible ciphertexts, while the \textit{keyspace} is the set of all possible keys.

There are two kinds of cryptosystems, \textit{symmetric} and \textit{asymmetric}. In symmetric-key cryptosystems, the same key is used for both encryption and decryption. As a consequence, symmetric-key cryptosystems must be implemented with a secure key exchange protocol, so that both the sender and receiver have access to the same key. Prominent examples of symmetric-key cryptosystems are the Data Encryption Standard (DES), and its replacement, the Advanced Encryption Standard (AES).

On the other hand, asymmetric-key (or public-key) cryptosystems use separate keys for encryption and decryption. The encryption key (also called the \textit{public key}) is shared publicly, while the decryption key (also called the \textit{private key}) is kept secret. Since the public key and private keys are different, there is no need to agree upon secure key exchange protocols. Usually, the security of asymmetric-key cryptosystems relies on the intractability of certain computational problems, for example, the RSA cryptosystem depends on the difficulty of big integer factorization~\cite{rivest_method_1978}, while the ElGamal cryptosystem depends on the difficulty of the discrete logarithm problem~\cite{blakley_public_1985}.

Homomorphic cryptosystems are a special type of cryptosystem in which operations can be securely performed on encrypted data. Suppose that for a public-key cryptosystem, $\mathcal{E}_k \left(p \right)$ is the encryption function using the public key $k \in \mathcal{K}$, and $\mathcal{D}_l \left(c \right)$ be the decryption function using the private key $l$. A cryptosystem is said to be homomorphic if its encryption function is homomorphic, that is, if it satisfies the relation
\begin{equation}
    \mathcal{E}_k \left(p_1 \boxplus p_2\right) = \mathcal{E}_k \left(p_1\right) \boxdot \mathcal{E}_k \left(p_2\right)
\end{equation}
where $p_1, p_2 \in \mathcal{P}$ are the plaintexts, and $\boxplus$ and $\boxdot$ are operations in $\mathcal{P}$ and $\mathcal{C}$ respectively~\cite{fontaine_survey_2007}. Furthermore, a homomorphic cryptosystem also satisfies~\cite{li_elliptic_2012}
\begin{equation}
    p_1 \boxplus p_2 = \mathcal{D}_l \left( \mathcal{E}_k \left(p_1\right) \boxdot \mathcal{E}_k \left(p_2\right) \right).
\end{equation}
In other words, a homomorphic cryptosystem preserves the operations that can be done with the plaintext without requiring an intermediary step of decrypting the ciphertext beforehand. It is important to note that the operations $\boxplus$ and $\boxdot$ need not be the same. A simple operation in the plaintext space may require a computationally intensive operation in the ciphertext space.

Homomorphic cryptosystems can be classified according to the plaintext and ciphertext operations, $\boxplus$ and $\boxdot$. If the plaintext operation $\boxplus$ is addition, the cryptosystem is said to be \textit{additively homomorphic}, and the cryptosystem is said to be \textit{multiplicatively homomorphic} if the plaintext operation $\boxplus$ is multiplication.

%Homomorphic cryptosystems
%Some examples of homomorphic cryptosystems include the Goldwasser--Micali cryptosystem~\cite{goldwasser_probabilistic_1984} and the Paillier cryptosystem~\cite{stern_public-key_1999}. Even some classic public-key cryptosystems are homomorphic, both RSA and ElGamal are multiplicatively homomorphic, while elliptic curve cryptosystems are additively homomorphic~\cite{li_elliptic_2012}. These cryptosystems, while relatively efficient, only support a limited set of operations on the encrypted data.

%However, there exists \textit{fully homomorphic} cryptosystems, which are not limited to a fixed set of operations, but allow any arbitrary operations on the ciphertext. The first  fully homomorphic cryptosystem was presented by Gentry in 2009. Gentry's cryptosystem, which uses lattice-based cryptography, allows the computation of computation of arbitrary Boolean circuits on binary data~\cite{gentry_fully_2009, shortell_secure_2016}. New algorithms based on Gentry's initial ideas have since been made~\cite{ sen_homomorphic_2013}. One notable example is fully homomorphic cryptosystem by Smart and Vercauteren,~\cite{hutchison_fully_2010}, later improved by Gentry and Halevi ~\cite{hutchison_implementing_2011} which relies on cyclotomic number fields. In 2016, Dasgupta and Pal proposed a fully homomorphic cryptosystem based on polynomial rings~\cite{dasgupta_design_2016}. These two cryptosytems are examples of fully homomorphic cryptosystems which have significantly simpler encryption, decryption, addition and multiplication algorithms.

\section{Partially Homomorphic Cryptosystems}

In this section, we will present two classical partially homomorphic cryptosystems (PHEs) which we will be testing in the study: the Paillier cryptosystem and the DGK cryptosystem. Both of these cryptosystems encrypt integer plaintexts and are additively homomorphic and allow for the multiplication of a plaintext scalar. We will then show how these PHEs may be adapted for secure floating-point computation. 

\subsection{The Paillier Cryptosystem}
The Paillier cryptosystem \cite{stern_public-key_1999}, developed by Pascal Paillier, is a probabilistic encryption scheme which is based on the composite residuosity class problem.

\subsubsection{Key Generation}
We choose two large primes $p$ and $q$, and set $n = pq $ and $ \lambda = \mathrm{lcm}\left(p-1,q-1\right)$.
We then define $L\left(x\right)$ to be the largest integer $v$ greater than zero such that $x-1 \geq vn$.
Then we select an integer $g$ such that $\gcd\left(L\left(g^\lambda \bmod n^2\right), n\right) = 1$ and $0\leq g \leq n^2$.
We define the public key as $(g,n)$ and the private key as $(p,q)$.

\subsubsection{Encryption and Decryption}
The encryption function to encrypt a plaintext $m \in \mathbb{Z}_n$ given a public key $(g,n)$ is defined as
\begin{align*}
  E(m) = g^m \cdot r^n \mod{n^2},
\end{align*}
where $r$ is a random non-negative integer less than $n^2$.

The decryption function to decrypt a ciphertext $c \in \mathbb{Z}_{n^2}$ given a private key $(p,q)$ is defined as:
\begin{align*}
  D(c) = L(c^\lambda \bmod n^2) \times (L(g^\lambda \bmod n^2))^{-1} \mod n
\end{align*}

\subsubsection{Homomorphic Properties of the Paillier Cryptosystem}
The Paillier cryptosystem supports additive homomorphism as well as the multiplication of a plaintext scalar to an encrypted message. These operations are defined as follows.
For all $m_1,m_2 \in \mathbb{Z}_n$ and $k\in \mathbb{N}$, the following homomorphic properties hold.
\begin{align*}
  D(E(m_1)g^k\bmod n^2)=(m_1+k)\bmod n & \text{ (add a plaintext constant)}\\
  D(E(m_1)E(m_2)\bmod n^2)=(m_1+m_2)\bmod n & \text{ (ciphertext addition)}\\
  D(E(m_1)^k\bmod n^2)= km_1\bmod n & \text{ (multiply a plaintext constant)}
\end{align*}

\subsection{The DGK Cryptosystem}
The DGK cryptosystem was published by Damg{\aa}rd, Geisler, and Kr{\o}igaard in 2007 in an effort to create a secure integer comparison scheme \cite{pieprzyk_efficient_2007, cryptoeprint:2008:321} which is widely used in the literature \cite{veugen_improving_2012}.

\subsubsection{Key Generation}
We denote $k,t,\ell$ as security parameters of the scheme, where $k>t>\ell$.
Let $p,q$ be primes such that
we can choose two $t$-bit primes $v_p$ and $v_q$ such that $v_p | (p-1)$ and $v_q | (q-1)$, and a small prime $u$ such that $u | (p-1)$ and $u | (q-1)$.
We denote $n = pq$.
We choose $g$ to be an integer of order $uv_pv_q$ and $h$ to be of order $v_pv_q$.

The DGK cryptosystem encrypts plaintexts in $\mathbb{Z}_u$ to ciphertexts in $\mathbb{Z}_n^\ast$.

The public key is $(n,g,h,u)$ and the private key is $(p,q,v_p,v_q)$.

\subsubsection{Encryption and Decryption}
To encrypt a message $m \in \mathbb{Z}_u$, the encryption function is defined as:
\begin{align*}
  E(m) = g^m \cdot h^r \mod{n},
\end{align*}
where $r$ is a random integer in $\mathbb{Z}_n$ which is longer than $2t$ bits.

To decrypt a ciphertext $c \in \mathbb{Z}_n^\ast$, decryption is achieved by first computing $c^{v_pv_q}$.
\begin{align*}
	c^{v_pv_q} \bmod n
	&= (g^m \cdot h^r)^{v_pv_q} \bmod n\\
	&= (g^{v_pv_q})^m \bmod n
\end{align*}
Since $(g^{v_pv_q})^m$ has order $u$, there is a one-to-one correspondence between plaintexts in $\mathbb{Z}_u$ and  $(g^{v_pv_q})^m$. A lookup table can thus be generated privately to successfully recover $m$.

\subsubsection{Homomorphic Properties of the DGK Cryptosystem}
The DGK cryptosystem supports additive homomorphism as well as the multiplication of a plaintext scalar to an encrypted message. These operations are defined as follows.
For all $m_1,m_2 \in \mathbb{Z}_u$ and $k\in \mathbb{N}$, the following homomorphic properties hold.
\begin{align*}
    D(E(m_1)g^k)=(m_1+k)\bmod u & \text{ (add a plaintext constant)}\\
    D(E(m_1)E(m_2))=(m_1+m_2)\bmod u & \text{ (ciphertext addition)}\\
    D(E(m_1)^k)= km_1\bmod u & \text{ (multiply a plaintext constant)}
\end{align*}

As the multiplicative homomorphism was not presented in the original paper, we provide a short proof here.
\begin{proof}
  Let $m \in \mathbb{Z}_u$ and $k\in \mathbb{N}$.
  We consider $E(m)^k = (g^m \cdot h^r \bmod{n})^k\bmod n$.
  \begin{align*}
    (g^m \cdot h^r \bmod{n})^k \bmod n
    &= (g^m \cdot h^r)^k \bmod{n}\\
    &= (g^m)^k \cdot (h^r)^k \bmod{n}\\
    &= g^{km} \cdot (h^{kr}) \bmod{n}
  \end{align*}
  Since $r$ is a random integer, $kr$ is also a random integer, Therefore, $g^{km} \cdot (h^{kr}) \bmod{n} = D(E(m)^k)$ is a valid encryption of the message $km$.
\end{proof}

\subsection{Floating-Point Arithmetic using PHEs}
\label{sec:fp_arithmetic}
We have discussed the Paillier and DGK cryptosystems, which encrypt and allow similar operations on encrypted integers. We will now show how these PHEs may be extended to a system which allows for secure floating-point computation.

In this section, we let $\oplus$ and $\otimes$ represent the homomorphic operations which correspond to the addition and multiplication of integers in a PHE, respectively. 

\subsubsection{Extension to Floating-Point Numbers}
\label{sec:fp_operations}
We can use the following protocol described in~\cite{ziad_cryptoimg:_2016} in order to extend Paillier and DGK to floating-point numbers.

We represent a floating-point (FP) number as a pair of two integers $(m,e)$ representing the mantissa and exponent of the FP number with respect to a base $b$. The mantissa $m$ is encrypted, while the exponent $e$ is unencrypted.
Let $a,b,c$ be FP numbers represented by the pairs $(m_a,e_a),(m_b,e_b),(m_c,e_c)$ respectively. We define the corresponding FP number operations as follows:
\begin{description}
  \item[Addition.]
    To compute $E\left(c\right)=E\left(a+b\right)$ we compute
	\begin{align*}
		E\left(m_c\right) &= 
		\begin{cases}
			E\left(m_a\right) \oplus \left(b^{e_b-e_a} \otimes E\left(m_b\right)\right) & \text{if } e_a \leq e_b \\
			E\left(m_b\right) \oplus \left(b^{e_a-e_b} \otimes E\left(m_a\right)\right) & \text{if } e_a > e_b
		\end{cases}, \\
		e_c &= 
		\begin{cases}
			e_a & \text{if } e_a \leq e_b \\
			e_b & \text{if } e_a > e_b
		\end{cases}.
	\end{align*}
  \item[Scalar multiplication.]
    To compute $E\left(c\right) = E\left(ab\right)$, where $a$ and $E\left(b\right)$ are known (i.e., $m_a$ is not encrypted), we compute
    \begin{align*}
      E\left(m_c\right) &= m_a \otimes E\left(m_b\right),\\
      e_c &= e_a + e_b.
    \end{align*}
\end{description}

\subsubsection{Secure Division}
We can use the following two-party scheme defined in~\cite{boukoros_lightweight_2017} to perform privacy-preserving division.

Suppose Bob has $E\left(x\right)$ and $E\left(y\right)$ and wants to obtain $E\left(x/y\right)$ without exposing the value of either variable.
\begin{itemize}
	\item Bob first selects a random non-zero number $r$ and computes $E\left(rx\right)$ and $E\left(ry\right)$. Bob can do this since $r$ is a plaintext constant.
	\item Bob sends $E\left(rx\right)$ and $E\left(ry\right)$ to Alice, who decrypts both values and computes $x/y$ in the plaintext domain.
	\item Alice encrypts and sends $E\left(x/y\right)$ to Bob.
\end{itemize}

\subsubsection{Secure Exponentiation}
\label{ssec:exponentiationprotocol}
 We can use the following two-party scheme, adapted from the protocol to calculate Euclidean distances used in \cite{hutchison_privacy-preserving_2009}, in order to perform secure exponentiation.

Suppose Alice encrypts an integer $x$ and sends it so Bob has $E\left(x\right)$, and wants to obtain $E\left(x^2\right)$ without exposing the value of $x$.
\begin{itemize}
	\item Bob first selects a random integer $r$ and computes $E\left(x+r\right)$. Bob can do this since $r$ is a plaintext constant.
	\item Bob sends $E\left(x+r\right)$ to Alice, who decrypts the ciphertext to obtain $x+r$.
	\item Alice squares $x+r$ and encrypts the result. She sends $E\left(\left(x+r\right)^2\right)$ to Bob.
	\item Bob computes $E\left(-2rx + r^2\right)$. He then computes
	\begin{align*}
		E\left(\left(x+r\right)^2\right)E\left(-2rx + r^2\right) = E\left(x^2\right).
	\end{align*}
\end{itemize}

\subsubsection{Secure Multiplication}
We can use the secure squaring protocol to arrive at a secure multiplication protocol, which then allows for the evaluation of polynomials.
Suppose Alice encrypts integers $x$ and $y$, and Bob has $E\left(x\right), E\left(y\right)$ and wants to obtain $E\left(xy\right)$.
\begin{itemize}
	\item Bob acquires $E\left(x^2\right)$ and $E\left(y^2\right)$ using the secure squaring protocol.
	\item Bob sends $E\left(x+y\right)$ to Alice, who decrypts the ciphertext to obtain $x+y$.
	\item Alice sends $E\left(\left(x+y\right)^2\right)$ to Bob.
	\item Bob then computes
	\begin{align*}
		\frac{1}{2}E\left(\left(x+y\right)^2\right)E\left(x^2\right)^{-1}E\left(y^2\right)^{-1} &= \frac{1}{2}E\left(\left(x+y\right)^2 - x^2 - y^2\right)\\
		&= \frac{1}{2}E\left(2xy\right)\\
		&= E\left(xy\right).
	\end{align*}
\end{itemize}

By applying these extensions to the Paillier and DGK cryptosystems, privacy-preserving floating-point arithmetic can be acheived. 

\section{Fully Homomorphic Cryptosystems}

We will also be testing fully homomorphic encryption (FHEs), which allow aribitrary computation on encrypted data.

The first ever fully homomorphic cryptosystem was invented by Gentry \cite{gentry_fully_2009}, which operates over ideal lattices. Gentry's original construction showed that FHE are possible, although they are known to be significantly slower than PHE.
Improving the efficiency of FHis an active area of research \cite{sen_homomorphic_2013}.

We will present two FHEs, the DGHV cryptosystem and the BGV cryptosystem, which both have open source implementations availble primarily used in homomorphic cryptography research. Since the mathematical details of these fully homomorphic cryptosystems are not relevant to describing their potential uses in secure image operations, we will instead briefly describe important developments and implementations of each cryptosystem.

\subsection{The DGHV Cryptosystem}
In 2009, Dijk, Gentry, Halevi, and Vaikuntanathan created a fully homomorphic cryptosystem which operates primarily through elementary modular arithmetic, commonly called the DGHV cryptosystem \cite{cryptoeprint:2009:616}. This was further improved in 2011 by Coron, Naccache, and Tibouchi, \cite{cryptoeprint:2011:277, cryptoeprint:2011:440} to support shorter public keys, reducing the size of secure public keys from 12.5 GB to 2.5 MB.

The DGHV cryptosystem encrypts single bits as integer ciphertexts. XOR and AND binary operations on encrypted bits are achieved by modular addition and modular multiplication on the integer ciphertexts. When operations are performed on ciphertexts, noise accumulates in the ciphertexts, which may prevent successful decryption. To eliminate this noise, a \textit{refresh} or \textit{recrypt} procedure is used to reduce noise in a ciphertext. The refresh procedure for DGHV involves operating on a ciphertext with a \textit{recryption matrix}, which is more time-intensive than key generation, encryption, decryption, addition, and multiplication procedures.

A Sage 4.7.2 implementation of the improved DGHV cryptosystem by Coron, et al. \cite{cryptoeprint:2011:440} may be found at \texttt{https://github.com/coron/fhe}. For small security parameters,  initial experiments using this implementation have shown it takes 0.06 seconds for key generation, 0.05 seconds for encryption, and 0.41 seconds for the recrypt procedure. For large security parameters, key generation takes 10 minutes, encryption takes 7 minutes and 15 seconds, and the recrypt procedure takes 11 minutes and 34 seconds \cite{cryptoeprint:2011:440}. Since the DGHV cryptosystem encrypts single plaintext bits, we can expect that operations on 32-bit floating point numbers are time-intensive, even under small security parameters.

A more recent C++ implementation of the improved DGHV cryptosystem \cite{cryptoeprint:2011:440} using the GNU Multiple Precision Arithmetic Library may be found at \texttt{https://github.com/deevashwer/Fully-Homomorphic-DGHV-and-Variants}. This implementation is more accessible than the original Sage 4.7.2 implementation, since the original code has compatibility issues with current versions of SageMath.

\subsection{HElib and the BGV Cryptosystem}
The BGV (Brakerski--Gentry--Vaikuntanathan) cryptosystem \cite{cryptoeprint:2011:277} is a fully homomorphic cryptosystem created based on the ring-learning with error problem. The BGV cryptosystem was constructed to surpass the limitations of prior fully homomorphic cryptosystems, which were based on the first fully homomorphic cryptosystem by Gentry \cite{gentry_fully_2009}.

One prominent implementation is \textit{HElib} (\texttt{https://github.com/shaih/HElib}) \cite{garay_algorithms_2014}, an open-source library which implements the BGV cryptosystem C++ using the NTL mathematical library, with optimizations to improve efficiency. Evaluation of operations on 120 inputs in HELib was performed in around 4 minutes, with an average of 2 seconds to process a single input \cite{hutchison_fully_2010,cryptoeprint:2011:566}.

The \textit{HElib} library has been adapted to Python using the Pyfhel library \cite{pyfhel_2018} maintained by Ibarrondo, Laurent (SAP) and Onen (EURECOM), and licensed under the GNU GPL v3 license. The Pyfhel library is a Python API for the HElib library, which supports the following operations on vectors/scalars of integers or binary ciphertexts:
\begin{itemize}
	\item Arithmetic operations: addition, subtraction, multiplication;
	\item Binary operations: AND, OR, NOT, XOR.
\end{itemize}

%facial recognition - implementation
\section{Related Work and Previous Implementations}

%% THIS SECTION IS MARKED FOR DELETION / AMPUTATION
% CryptoImg

% example of homomorphic encryption / image manipulation past work

% minor limitation: improvement on previous work, but not a direct comparison

% major limitation: does not discuss security: are modified images also secure?

%There has been work done regarding the application of homomorphic cryptosystems in image processing. In particular, Ziad, et al. introduced a library called \textit{CryptoImg} that uses the homomorphic properties of the Paillier cryptosystem to apply image operations securely \cite{ziad_cryptoimg:_2016}. This shows that it is indeed possible to do various image operations in a homomorphic cryptosystem. Ziad, et al. also showed experimental results establishing the slow performance of image operations uner a homomorphic cryptosystem. For instance, while sharpening and applying a Sobel filter each take less than a second when applied to a $512\times 512$ plaintext image, when applied to an encrypted image, sharpening required at least 238.257 seconds, and applying the Sobel filter required at least 147.567 seconds \cite{ziad_cryptoimg:_2016}.

%However, a major limitation of \textit{CryptoImg} is that it does not consider image security. Even though the authors have established that the Paillier cryptosystem itself cannot be broken \cite{ziad_cryptoimg:_2016}, we believe that a cryptanalysis of the encrypted images after they have been operated is necessary, since image operations may be considered additional information in a known plaintext attack.

%Furthermore, Ziad, et al. only presented a visual comparison and evaluation to establish the quality of the recovered images. We believe that additional image quality benchmarks, such as those presented in \cite{ahmed_benchmark_2016, ahmad_efficiency_2012}, would allow for quantitative comparisons of image quality.

%This study can be further improved by also considering the use of fully homomorphic cryptosystems such as the one presented by Dasgupta and Pal \cite{dasgupta_design_2016}, and the fully homomorphic cryptosystem introduced by Smart and Vercauteren \cite{hutchison_fully_2010} and improved by Gentry and Halevi  \cite{hutchison_implementing_2011}. These fully homomorphic cryptosytems have yet to be implemented for image processing operations.

% HElib
%Another related work would be the implementation of a fully homomorphic encryption. Halevi and Shoup \cite{garay_algorithms_2014} introduced \textit{HElib}, a library that implements the Brakerski--Gentry--Vaikuntanathan (BGV) homomorphic cryptosystem. This library also makes use of various optimizations to speed up the homomorphic operations, due to homomorphic cryptosystems being slower than other cryptosystems \cite{sen_homomorphic_2013}.

Erkin, et al. \cite{hutchison_privacy-preserving_2009-2} devised a method to incorporate the use of homomorphic cryptosystems into the eigenfaces method. In their study, they proposed a two-party system where Alice holds an encrypted image $\left[\Gamma\right]$, while Bob maintains a database of $K$ eigenfaces $\mathbf{u}_1, \ldots, \mathbf{u}_K$, and feature vectors $\Omega_1, \ldots, \Omega_M$ in the clear.

\begin{figure}[!h]
    \centering
    \includegraphics[width=7cm]{figures/secure_eigenfaces.png}
    \caption{A diagram of the process \cite{hutchison_privacy-preserving_2009-2}.}
\end{figure}


In order to ensure privacy, the steps in the eigenfaces algorithm, namely: projection, distance computation, and match finding, are done within the encrypted domain, i.e., using the operations in the Paillier cryptosystem. 

Projection is similar to that of the original eigenfaces algorithm, except the operations are replaced with their respective operations in the cryptosystem. Distance computation in this version is somewhat different from the original eigenfaces method, in that it deals with the square of the Euclidean distance since the relative order of the distances is only important when comparing these during the match finding step \cite{hutchison_privacy-preserving_2009-2}.
\begin{align*}
    d_i &= \left\lVert \Omega_i - \bar{\Omega} \right\rVert ^2 = \sum_{j=1}^{K} \left(\omega_{ij} - \bar{\omega}_j\right)^2 \\
        &= \underbrace{\sum_{j=1}^{K} \omega_{ij}^2}_{\mathcal{S}_1} + \underbrace{\sum_{j=1}^{K} \left(-2 \omega_{ij} \bar{\omega_j}\right)}_{\mathcal{S}_2} + \underbrace{\sum_{j=1}^{K} \bar{\omega}_{j}^2}_{\mathcal{S}_3} 
\end{align*}

Computing for the distances within Paillier would just be multiplying the encrypted sums together.
\[
\left[d_i\right] = \left[\mathcal{S}_1\right] \cdot \left[\mathcal{S}_2\right] \cdot \left[\mathcal{S}_3\right]
\]

The terms $\left[\mathcal{S}_1\right]$ and $\left[\mathcal{S}_2\right]$ can be easily computed by Bob, since he already knows both $\omega_i$ in the clear and $\left[\bar{\omega}_i\right]$ which is in encrypted form. Computing for $\left[\mathcal{S}_3\right]$ is trickier because Bob cannot compute for $\left[\bar{\omega}_i^2\right]$ because pairwise multiplication is not supported in Paillier, that is why Bob needs help from Alice to square a number through a protocol described below \cite{hutchison_privacy-preserving_2009-2}.

Before Bob sends $\left[\bar{\omega}_i\right]$ to Alice for squaring, he adds a random number $r_i$ to compute $\left[x_i\right] = \left[\bar{\omega}_i + r_i\right] = \left[\bar{\omega}_i\right] \cdot \left[r_i\right]$, where $r_i$ is obviously distinct for every $i$, then sends $\left[x_i\right]$ to Alice. She then decrypts it and computes $x_i^2$, and then computes $\mathcal{S}_3^\prime = \sum_{i=1}^{K} x_i^2$, after which she encrypts the sum and sends $\left[\mathcal{S}_3^\prime\right]$ to Bob. Now, he can compute for $\left[\mathcal{S}_3\right]$ as follows:
\[
\left[\mathcal{S}_3\right] = \left[\mathcal{S}_3^\prime\right] \cdot \prod_{j=1}^{K} \left(\left[\bar{\omega}_j\right]^{-2r_j} \cdot \left[-r_j^2\right]\right)
\]

The protocol for squaring a number as described earlier is a workaround for the limitations of Paillier or any other partially homomorphic cryptosystem that only supports addition and scalar multiplication. This can be possibly extended so that operations such as pairwise multiplication and exponentiation are also supported, provided that the protocol involves two parties. 

The match finding step is done by comparing the distances obtained from the previous step to a specified threshold $T$. If the minimum distance is smaller than $T$, then a match is found, and the encrypted identity of the match is returned to Alice \cite{hutchison_privacy-preserving_2009-2}. Getting the minimum among the encrypted distances would involve comparing two encrypted numbers, which Paillier does not support. Instead, another homomorphic cryptosystem by Damg{\aa}rd, Geisler, and Kr{\o}igaard \cite{pieprzyk_efficient_2007} is used for the comparison protocol.



\section{Summary}
We have seen how various image processing operations make use of a series of additions and multiplications. Because of this, it is possible to use a homomorphic cryptosystem in order to apply image operations directly on encrypted images, as demonstrated by the \textit{CryptoImg} library.

In particular, we aim to address the shortcomings of \textit{CryptoImg} by developing a library to test various homomorphic encryption schemes with the help of the \textit{HElib} library to be able to implement fully homomorphic encryption schemes efficiently. This helps us determine the practicality of using homomorphic cryptosystems in image processing, especially when dealing with more complex image operations.
