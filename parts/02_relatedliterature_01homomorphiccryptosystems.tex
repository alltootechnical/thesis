\section{Homomorphic Cryptosystems}
% [include mention of metrics by which they are compared]

In cryptography, a cryptosystem consists of an encryption function $\mathcal{E}$ and a decryption function $\mathcal{D}$, along with the plaintext space $\mathcal{P}$, ciphertext space $\mathcal{C}$ and the key space $\mathcal{K}$~\cite{bauer_cryptosystem_2005}. A \textit{plaintext} is text that can be commonly understood, while a \textit{ciphertext} results from encrypting the plaintext using an encryption key. The \textit{plaintext space} is the set of all possible plaintexts, the \textit{ciphertext space} is the set of all possible ciphertexts, while the \textit{keyspace} is the set of all possible keys.

There are two kinds of cryptosystems, \textit{symmetric} and \textit{asymmetric}. In symmetric-key cryptosystems, the same key is used for both encryption and decryption. As a consequence, symmetric-key cryptosystems must be implemented with a secure key exchange protocol, so that both the sender and receiver have access to the same key. Prominent examples of symmetric-key cryptosystems are the Data Encryption Standard (DES), and its replacement, the Advanced Encryption Standard (AES).

On the other hand, asymmetric-key (or public-key) cryptosystems use separate keys for encryption and decryption. The encryption key (also called the \textit{public key}) is shared to everybody, while the decryption key (also called the \textit{private key}) is kept secret. Since the public key and private keys are different, there is no need to agree upon secure key exchange protocols. Usually, the security of asymmetric-key cryptosystems relies on the intractability of certain computational problems, for example, the RSA cryptosystem depends on the difficulty of big integer factorization~\cite{rivest_method_1978}, while the ElGamal cryptosystem depends on the difficulty of the discrete logarithm problem~\cite{blakley_public_1985}.

Homomorphic cryptosystems are a special type of cryptosystem in which operations can be securely performed on encrypted data. Suppose that for a public-key cryptosystem, $\mathcal{E}_k \left(p \right)$ is the encryption function using the public key $k \in \mathcal{K}$, and $\mathcal{D}_l \left(c \right)$ be the decryption function using the private key $l$. A cryptosystem is said to be homomorphic if its encryption function is homomorphic, that is, if it satisfies the relation
\begin{equation}
    \mathcal{E}_k \left(p_1 \boxplus p_2\right) = \mathcal{E}_k \left(p_1\right) \boxdot \mathcal{E}_k \left(p_2\right)
\end{equation}
where $p_1, p_2 \in \mathcal{P}$ are the plaintexts, and $\boxplus$ and $\boxdot$ are operations in $\mathcal{P}$ and $\mathcal{C}$ respectively~\cite{fontaine_survey_2007}. Furthermore, a homomorphic cryptosystem also satisfies~\cite{li_elliptic_2012}
\begin{equation}
    p_1 \boxplus p_2 = \mathcal{D}_l \left( \mathcal{E}_k \left(p_1\right) \boxdot \mathcal{E}_k \left(p_2\right) \right).
\end{equation}
In other words, a homomorphic cryptosystem preserves the operations that can be done with the plaintext without requiring an intermediary step of decrypting the ciphertext beforehand. It is important to note that the operations $\boxplus$ and $\boxdot$ need not be the same. A simple operation in the plaintext space may require a computationally intensive operation in the ciphertext space.

Homomorphic cryptosystems are classified according to the plaintext and ciphertext operations, $\boxplus$ and $\boxdot$. If the plaintext operation $\boxplus$ is addition, the cryptosystem is said to be \textit{additively homomorphic}, and the cryptosystem is said to be \textit{multiplicatively homomorphic} if the plaintext operation $\boxplus$ is multiplication. Some examples of homomorphic cryptosystems include the Goldwasser--Micali cryptosystem~\cite{goldwasser_probabilistic_1984} and the Paillier cryptosystem~\cite{stern_public-key_1999}. Even some classic public-key cryptosystems are homomorphic, both RSA and ElGamal are multiplicatively homomorphic, while elliptic curve cryptosystems are additively homomorphic~\cite{li_elliptic_2012}. These cryptosystems, while relatively efficient, only support a limited set of operations on the encrypted data.

However, there exists \textit{fully homomorphic} cryptosystems, which are not limited to a fixed set of operations, but allow any arbitrary operations on the ciphertext. The first  fully homomorphic cryptosystem was presented by Gentry in 2009. Gentry's cryptosystem, which uses lattice-based cryptography, allows the computation of computation of arbitrary Boolean circuits on binary data~\cite{gentry_fully_2009, shortell_secure_2016}. New algorithms based on Gentry's initial ideas have since been made~\cite{ sen_homomorphic_2013}. One notable example is fully homomorphic cryptosystem by Smart and Vercauteren,~\cite{hutchison_fully_2010}, later improved by Gentry and Halevi ~\cite{hutchison_implementing_2011} which relies on cyclotomic number fields. In 2016, Dasgupta and Pal proposed a fully homomorphic cryptosystem based on polynomial rings~\cite{dasgupta_design_2016}. These two cryptosytems are examples of fully homomorphic cryptosystems which have significantly simpler encryption, decryption, addition and multiplication algorithms.
