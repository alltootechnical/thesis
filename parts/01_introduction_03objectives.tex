\section{Research Objectives}

Previous attempts in the literature to provide privacy-preserving image processing tasks have primarily used the Paillier and  Damg{\aa}rd--Geisler--Kr{\o}igaard (DGK) cryptosystems to perform linear operations on encrypted data~\cite{ziad_cryptoimg:_2016,hutchison_privacy-preserving_2009}. In this study, we extend this research to provide an objective comparison of various homomorphic cryptosystems currently in the literature in performing non-linear image operations.

Our main research objective is the assessment of the practicality of various homomorphic cryptosystems with regard to image processing on faces. Homomorphic cryptosystems are known to be slower in terms of encryption, operation, and decryption time~\cite{sen_homomorphic_2013}. Furthermore, the operations on encrypted data supported by homomorphic cryptosystems are limited, and some schemes require preprocessing to be performed on images, limiting the range of operations which can be performed efficiently~\cite{li_elliptic_2012}.

In summary, in this study, we 
\begin{enumerate}
    \item Implement common image processing operations using known homomorphic encryption schemes, then determine which schemes admit more image processing operations and extend known cryptosystems to support more operations, if possible;
    \item Compare the time efficiency and accuracy of common image processing operations under known homomorphic encryption schemes, and of common image processing operations without using homomorphic encryption;
    \item Create a software library to allow for convenient use and comparison of these homomorphic encryption schemes for image processing and facial recognition.
\end{enumerate}
