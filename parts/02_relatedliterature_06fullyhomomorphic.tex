\section{Fully Homomorphic Cryptosystems}

We will also be testing fully homomorphic encryption (FHEs), which allow aribitrary computation on encrypted data.

The first ever fully homomorphic cryptosystem was invented by Gentry \cite{gentry_fully_2009}, which operates over ideal lattices. Gentry's original construction showed that FHE are possible, although they are known to be significantly slower than PHE. 
Improving the efficiency of FHis an active area of research \cite{sen_homomorphic_2013}. 

We will present two FHEs, the DGHV cryptosystem and the BGV cryptosystem, which both have open source implementations availble primarily used in homomorphic cryptography research. Since the mathematical details of these fully homomorphic cryptosystems are not relevant to describing their potential uses in secure image operations, we will instead briefly describe important developments and implementations of each cryptosystem.

<<<<<<< HEAD
Given a message $m \in \mathbb{Z}$, we denote
\begin{align*}
		m_p(x) = a_0 + a_1x + a_2x^2 + \cdots + a_kx^k
\end{align*}
where $a_k a_{k-1}\cdots a_2a_1_0$ is the binary representation of $m$.
=======
\subsection{The DGHV Cryptosystem}
In 2009, Dijk, Gentry, Halevi, and Vaikuntanathan created a fully homomorphic cryptosystem which operates primarily through elementary modular arithmetic, commonly called the DGHV cryptosystem \cite{cryptoeprint:2009:616}. This was further improved in 2011 by Coron, Naccache, and Tibouchi, \cite{cryptoeprint:2011:277, cryptoeprint:2011:440} to support shorter public keys, reducing the size of secure public keys from 12.5 GB to 2.5 MB. 
>>>>>>> 21e97430400d95bf5a42495d659f103ee6363cdc

The DGHV cryptosystem encrypts single bits as integer ciphertexts. XOR and AND binary operations on encrypted bits are achieved by modular addition and modular multiplication on the integer ciphertexts. When operations are performed on ciphertexts, noise accumulates in the ciphertexts, which may prevent successful decryption. To eliminate this noise, a \textit{refresh} or \textit{recrypt} procedure is used to reduce noise in a ciphertext. The refresh procedure for DGHV involves operating on a ciphertext with a \textit{recryption matrix}, which is more time-intensive than key generation, encryption, decryption, addition, and multiplication procedures.

A Sage 4.7.2 implementation of the improved DGHV cryptosystem by Coron, et al. \cite{cryptoeprint:2011:440} may be found at \texttt{https://github.com/coron/fhe}. For small security parameters,  initial experiments using this implementation have shown it takes 0.06 seconds for key generation, 0.05 seconds for encryption, and 0.41 seconds for the recrypt procedure. For large security parameters, key generation takes 10 minutes, encryption takes 7 minutes and 15 seconds, and the recrypt procedure takes 11 minutes and 34 seconds \cite{cryptoeprint:2011:440}. Since the DGHV cryptosystem encrypts single plaintext bits, we can expect that operations on 32-bit floating point numbers are time-intensive, even under small security parameters.

A more recent C++ implementation of the improved DGHV cryptosystem \cite{cryptoeprint:2011:440} using the GNU Multiple Precision Arithmetic Library may be found at \texttt{https://github.com/deevashwer/Fully-Homomorphic-DGHV-and-Variants}. This implementation is more accessible than the original Sage 4.7.2 implementation, since the original code has compatibility issues with current versions of SageMath.

\subsection{HElib and the BGV Cryptosystem}
The BGV (Brakerski--Gentry--Vaikuntanathan) cryptosystem \cite{cryptoeprint:2011:277} is a fully homomorphic cryptosystem created based on the ring-learning with error problem. The BGV cryptosystem was constructed to surpass the limitations of prior fully homomorphic cryptosystems, which were based on the first fully homomorphic cryptosystem by Gentry \cite{gentry_fully_2009}.

One prominent implementation is \textit{HElib} (\texttt{https://github.com/shaih/HElib}) \cite{garay_algorithms_2014}, an open-source library which implements the BGV cryptosystem C++ using the NTL mathematical library, with optimizations to improve efficiency. Evaluation of operations on 120 inputs in HELib was performed in around 4 minutes, with an average of 2 seconds to process a single input \cite{hutchison_fully_2010,cryptoeprint:2011:566}.

The \textit{HElib} library has been adapted to Python using the Pyfhel library \cite{pyfhel_2018} maintained by Ibarrondo, Laurent (SAP) and Onen (EURECOM), and licensed under the GNU GPL v3 license. The Pyfhel library is a Python API for the HElib library, which supports the following operations on vectors/scalars of integers or binary ciphertexts:
\begin{itemize}
	\item Arithmetic operations: addition, subtraction, multiplication;
	\item Binary operations: AND, OR, NOT, XOR.
\end{itemize}
