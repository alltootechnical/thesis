\section{Fully homomorphic cryptosystems}
\subsection{Dasgupta-Pal cryptosystem}
The Dasgupta-Pal cryptosystem is a fully homomorphic cryptosystem proposed in 2016 by Smaranika Dasgupta and S. K. Pal \cite{dasgupta_design_2016}, which encrypts integer plaintexts in $\mathbb{Z}_n$ to polynomial ciphertexts. We begin our presentation of the Dasgupta-Pal cryptosystem with the following definition.

Given a message $m \in \mathbb{n}$, we denote
\begin{align*}
		m_p(x) = a_0 + a_1x + a_2x^2 + \cdots + a_kx^k
\end{align*}
where $a_ka_{k-1}\cdots a_2a_1_0$ is the binary representation of $m$.
\subsubsection{Dasgupta-Pal cryptosystem description}
Let $\ell$ denote the security parameter of the cryptosystem.
Let $S_k$ be a prime number with $\ell$ bits.
Choose a random even integer $z$ of length $\log_2{\ell}$.

Let the secret key be $S_k$, and let the refresh key be $R_k = z \cdot S_k$. The secret key is kept private to the encrypting/decrypting parties, while the refresh key is made publicly available.

The scheme defines the encryption algorithm for a message $m \in \mathbb{Z}_n$ as follows:
\begin{align*}
	E(m) = y(x) + S_k\times d(x)
\end{align*}
where
$y(x)$ is a polynomial of degree $n$ such that $m_p(x) \equiv y(x) \bmod S_k$ and $d(x)$ is a randomly chosen polynomial of degree $n$.

Furthermore, the decryption algorithm to recover $m_p(x)$ from a ciphertext polynomial $c(x)$ as
\begin{align*}
	m_p(x) = c(x) \bmod S_k \bmod 2
\end{align*}
Thus $D(c(x))$ is defined as the integer recovered from the coefficients of the polynomial $c(x) \bmod S_k \bmod 2$

As a fully homomorphic cryptosystem, homomorphic operations on ciphertext introduce noise which may interfere with decryption.
To eliminate noise from a polynomial ciphertext, the following refresh function is used.
\begin{align*}
	R(c(x)) = c(x) \bmod R_k
\end{align*}

\subsubsection{Homomorphic properties}
It has been shown \cite{dasgupta_design_2016} that the following properties hold for all integer messages $m_1, m_2 \in \mathbb{Z}_n$ in the Dasgupta-Pal cryptosystem.
\begin{description}
  \item[Additive homomorphism]
  \begin{align*}
    D(E(m_1) + E(m_2))= m_1+m_2 & \text{ (used for binary addition)}
  \end{align*}
  \item[Multiplicative homomorphism]
  \begin{align*}
    D(E(m_1) \times E(m_2))= m_1 \times m_2 & \text{ (used for binary multiplication)}
  \end{align*}
\end{description}

\section{}
