\section{Fully Homomorphic Cryptosystems}

We will also be testing fully homomorphic cryptosytems (FHEs), which allow aribitrary computation on encrypted data.

The first ever fully homomorphic cryptosystem was invented by Gentry \cite{gentry_fully_2009}, which operates over ideal lattices. Gentry's original construction showed that fully homomorphic cryptosystems are possible, although they are known to be significantly slower than partially homomorphic cryptosytems. 
Improving the efficiency of fully homomorphic cryptosytems is an active area of research \cite{sen_homomorphic_2013}. 

We will present two FHEs, the DGHV cryptosystem and the BGV cryptosystem, which both have open source implementations availble primarily used in homomorphic cryptography research. Since the mathematical details of these fully homomorphic cryptosystems are not relevant to describing their potential uses in secure image operations, we will instead briefly describe pertinent developements and implementations of each cryptosystem.

\subsection{DGHV Cryptosystem}
In 2009, Dijk, Gentry, Halevi, and Vaikuntanathan created a fully homomorphic cryptosystem which operates primarily through elementary modular arithmetic, commonly called the DGHV cryptosystem \cite{cryptoeprint:2009:616}. This was further improved in 2011 \cite{cryptoeprint:2011:277, cryptoeprint:2011:440} to support shorter public keys, reducing the size of secure public keys from 12.5 GB to 2.5 MB. 

The DGHV cryptosystem encrypts single bits as integer ciphertexts. XOR and AND binary operations on encrypted bits are achieved by modular addition and modular multiplication on the integer ciphertexts. When operations are performed on ciphertexts, noise accumulates in the ciphertexts, which may prevent successful decryption.



A C++ implementation of the improved DGHV cryptosystem \cite{cryptoeprint:2011:440} using the GNU Multiple Precision Arithmetic Library may be found at https://github.com/deevashwer/Fully-Homomorphic-DGHV-and-Variants.

For small security parameters, initial experiments using the DGHV cryptosystem have shown it takes 4.38 seconds for key generation, and 

\subsection{HElib and the BGV Cryptosystem}
The BGV (Brakerski--Gentry--Vaikuntanathan) cryptosystem \cite{cryptoeprint:2011:277} is a fully homomorphic cryptosystem created based on the ring-learning with error problem. The BGV cryptosystem was constructed to surpass the limitations of prior fully homomorphic cryptosystems, which were based on the first fully homomorphic cryptosystem by Gentry \cite{gentry_fully_2009}.

For this study, we will be using \textit{HElib} \cite{garay_algorithms_2014}, an open-source library which implements the BGV cryptosystem with optimizations to improve efficiency. Evaluation of operations on 120 inputs in HELib was performed in around 4 minutes, with an average of 2 seconds to process a single input \cite{hutchison_fully_2010,cryptoeprint:2011:566}.

The \textit{HElib} library has been adapted to Python using the Pyfhel library \cite{pyfhel_2018} maintained by Ibarrondo, Laurent (SAP) and Onen (EURECOM), and licensed under the GNU GPL v3 license. The Pyfhel library is a Python API for the HElib library, which supports the following operations on vectors/scalars of integers or binary ciphertexts:
\begin{itemize}
	\item Arithmetic operations: addition, subtraction, multiplication;
	\item Binary operations: AND, OR, NOT, XOR.
\end{itemize}
