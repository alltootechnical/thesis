\section{Fully Homomorphic Cryptosystems}

We will also be testing fully homomorphic cryptosytems, which allow aribitrary computation on encrypted data.
While generally being more flexible, fully homomorphic cryptosytems are known to be significantly slower than partially homomorphic cryptosytems; improving the efficiency of fully homomorphic cryptosytems is an active area of research \cite{sen_homomorphic_2013}.

We will present two such cryptosystems, the Dasgupta--Pal cryptosystem, which we present in full to explain a correction we propose to it, and the BGV cryptosystem, which has open source implementations availble primarily used in homomorphic cryptography research.

\subsection{The Dasgupta--Pal Cryptosystem}
The Dasgupta--Pal cryptosystem is a fully homomorphic cryptosystem proposed in 2016 by Smaranika Dasgupta and S. K. Pal \cite{dasgupta_design_2016}, which encrypts integer plaintexts in $\mathbb{Z}_n$ to polynomial ciphertexts. We begin our presentation of the Dasgupta--Pal cryptosystem with the following definition.

Given a message $m \in \mathbb{Z}$, we denote
\begin{align*}
		m_p(x) = a_0 + a_1x + a_2x^2 + \cdots + a_kx^k
\end{align*}
where $a_ka_{k-1}\cdots a_2a_1_0$ is the binary representation of $m$.

In the original scheme presented in \cite{dasgupta_design_2016}, the secret key $S_k$ is set to be a large prime. In Appendix \ref{chap:correction}, we present a case where the Dasgupta--Pal cryptosystem fails and prove that setting the secret key $S_k = 2p$, where $p$ is a large prime, corrects errors in the Dasgupta--Pal cryptosystem.
We now decribe the correct cryptosystem.
\subsubsection{Corrected Dasgupta--Pal Cryptosystem Description}
Let $\ell$ denote the security parameter of the cryptosystem.
Let $S_k = 2p$, where $p$ is a prime number with $\ell - 1$ bits.
Choose a random even integer $z$ of length $\log_2{\ell}$.

Let the secret key be $S_k$, and let the refresh key be $R_k = z \cdot S_k$. The secret key is kept private to the encrypting/decrypting parties, while the refresh key is made publicly available.

The scheme defines the encryption algorithm for a message $m$ as follows:
\begin{align*}
	E(m) = y(x) + S_k\times d(x)
\end{align*}
where
$y(x)$ is a polynomial of degree $n$ such that $m_p(x) \equiv y(x) \bmod S_k$ and $d(x)$ is a randomly chosen polynomial of degree $n$. This encrypts each coefficient of of $m_p(x)$ by adding some multiple of $S_k$ to it. Through this process, each bit of the message is encrypted separately.

Furthermore, the decryption algorithm to recover $m_p(x)$ from a ciphertext polynomial $c(x)$ as
\begin{align*}
	m_p(x) = c(x) \bmod S_k \bmod 2.
\end{align*}
Thus $D(c(x))$ is defined as the integer recovered from the coefficients of the polynomial $c(x) \bmod S_k \bmod 2$.

Dasgupta and Pal note that homomorphic operations on ciphertext introduce noise which may interfere with decryption, due to the potential increase in ciphertext values.
To eliminate noise from a polynomial ciphertext, the following refresh function is used:
\begin{align*}
	R(c(x)) = c(x) \bmod R_k.
\end{align*}

\subsubsection{Homomorphic Properties of the Dasgupta--Pal Cryptosystem}
It has been shown \cite{dasgupta_design_2016} that the following properties hold for all integer messages $m_1, m_2 \in \mathbb{Z}_n$ in the Dasgupta--Pal cryptosystem. These properties arise since each coefficient in a ciphertext polynomial is essentially an independently encrypted bit of the original message.
\begin{description}
	\item[XOR on ciphertexts]
	A bitwise exclusive or (XOR) operation on integer messages can be achieved by adding the coefficients pairwise between two ciphertexts.
	\begin{align*}
		D(E(m_1)+E(m_2)) = m_1 \text{ XOR } m_2
	\end{align*}
	\item[AND on ciphertexts]
	Similarly, pair-wise multiplication corresponds to bitwise AND of integer messages.
	\begin{align*}
		D(E(a) \otimes E(b)) = a \text{ AND } b = ab \bmod 2
	\end{align*}
	where $\otimes$ denotes pairwise multiplication,
	\begin{align*}
		\sum_{i=1}^n{a_ix^i} \otimes \sum_{i=1}^n{b_ix^i} = \sum_{i=1}^n{a_ib_ix^i}.
	\end{align*}
\end{description}

Arbitrary computation on data can thus be achived using these bitwise operations. While the Dasgupta--Pal cryptosystem requires significant amounts of memory, as each bit of plaintext is encrypted into an integer, this approach provides a flexible method to perform relatively fast arbitrary computation on encrypted data. Encryption and decryption in the Dasgupta--Pal cryptosystem are accomplished using few arithmetic operations.

\subsection{HElib and the BGV Cryptosystem}
The BGV (Brakerski--Gentry--Vaikuntanathan) cryptosystem \cite{cryptoeprint:2011:277} is a fully homomorphic cryptosystem created based on the ring-learning with error problem. The BGV cryptosystem was constructed to surpass the limitations of prior fully homomorphic cryptosystems, which were based on the first fully homomorphic cryptosystem by Gentry \cite{gentry_fully_2009}.

The details regarding the construction of the BGV cryptosystem are beyond the scope of this paper. For this study, we will be using \textit{HElib} \cite{garay_algorithms_2014}, an open-source library which implements the BGV cryptosystem with optimizations to improve efficiency. Evaluation of operations on 120 inputs in HELib was performed in around 4 minutes, with an average of 2 seconds to process a single input \cite{hutchison_fully_2010,cryptoeprint:2011:566}.

The \textit{HElib} library has been adapted to Python using the Pyfhel library \cite{pyfhel_2018} maintained by Ibarrondo, Laurent (SAP) and Onen (EURECOM), and licensed under the GNU GPL v3 license. The Pyfhel library is a Python API for the HElib library, which supports the following operations on vectors/scalars of integers or binary ciphertexts:
\begin{itemize}
	\item Arithmetic operations: addition, subtraction, multiplication;
	\item Binary operations: AND, OR, NOT, XOR.
\end{itemize}
