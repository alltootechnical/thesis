\section{Homomorphic Cryptosystems}

A cryptosystem~\cite{bauer_cryptosystem_2005} consists of an encryption function $\mathcal{E}$ and a decryption function $\mathcal{D}$, which operate on three sets: the plaintext space $\mathcal{P}$, ciphertext space $\mathcal{C}$ and the key space $\mathcal{K}$ A \textit{plaintext}, an element of the \textit{plaintext space}, is text that can be commonly understood within a larger group. Given a \textit{key}, an element of the \textit{key space}, the encryption function maps a plaintext to some \textit{ciphertext}, an element of the ciphertext space which can only be understood by authorized parties. The decryption function similarly maps a ciphertext back to its corresponding plaintext, given an appropiate key. By sending data as ciphertext and only exposing the corresponding plaintexts to authorized parties who have access to the appropiate keys, secure data transmission can be achieved.

Cryptosystems may be classified as either \textit{symmetric} or \textit{asymmetric}. 
In symmetric-key cryptosystems, the same key is used for both encryption and decryption. As a consequence, symmetric-key cryptosystems must be implemented with a secure key exchange protocol, so that both the sender and receiver have access to the same key while keeping the key secret. 
Prominent examples of symmetric-key cryptosystems are the Data Encryption Standard (DES), and its replacement, the Advanced Encryption Standard (AES).

On the other hand, asymmetric-key (or public-key) cryptosystems use separate keys for encryption and decryption. The encryption key (also called the \textit{public key}) is shared publicly, while the decryption key (also called the \textit{private key}) is kept secret. Since the public key and private keys differ, secure key exchange protocols are not required. The security of asymmetric-key cryptosystems rely on the intractability of certain computational problems, for example, the RSA cryptosystem depends on the difficulty of big integer factorization~\cite{rivest_method_1978}, while the ElGamal cryptosystem depends on the difficulty of the discrete logarithm problem~\cite{blakley_public_1985}.

Homomorphic cryptosystems are a special type of cryptosystem in which operations can be securely performed on encrypted data. Suppose that in a public-key cryptosystem, $\mathcal{E}_k \left(p \right)$ is the encryption function which uses a public key $k$, and $\mathcal{D}_l \left(c \right)$ be the decryption function which uses a private key $l$. The cryptosystem is said to be homomorphic if its encryption function is homomorphic, that is, if it satisfies the relation
\begin{equation}
    \label{eq:homomorphic_definition_encryption}
    \mathcal{E}_k \left(p_1 \boxplus p_2\right) = \mathcal{E}_k \left(p_1\right) \boxdot \mathcal{E}_k \left(p_2\right)
\end{equation}
where $p_1, p_2 \in \mathcal{P}$ are plaintexts, and $\boxplus$ and $\boxdot$ are operations in $\mathcal{P}$ and $\mathcal{C}$ respectively~\cite{fontaine_survey_2007}. By applying the decryption function $\mathcal{D}_l$ to both sides of equation~\ref{eq:homomorphic_definition_encryption}, we can see that a homomorphic cryptosystem also satisfies~\cite{li_elliptic_2012}
\begin{equation}
    \label{eq:homomorphic_definition_decryption}
    p_1 \boxplus p_2 = \mathcal{D}_l \left( \mathcal{E}_k \left(p_1\right) \boxdot \mathcal{E}_k \left(p_2\right) \right).
\end{equation}
In other words, a homomorphic cryptosystem allows operations to be performed on recovered plaintexts by performing corresponding operations on ciphertexts. Using a homomorphic cryptosystem, secure computation can be achieved by encrypting plaintexts and then operating on the ciphertexts. We note that the operations $\boxplus$ and $\boxdot$ need not be the same. A simple operation in the plaintext space may require a computationally intensive operation in the ciphertext space.

Homomorphic cryptosystems can be classified according to the supported plaintext operation(s) $\boxplus$ which can be securely computed. The cryptosystem is said to be \textit{additively homomorphic} if the plaintext operation $\boxplus$ is addition, and the cryptosystem is said to be \textit{multiplicatively homomorphic} if the plaintext operation $\boxplus$ is multiplication. Homomorphic cryptosystems which only allow for a limited set of operations to be computed securely are called \textit{partially homomorphic} while those which allow for arbitrary operations to be computed securely are called \textit{fully homomorphic}. Numerous partially homomorphic and fully homomorphic cryptosystems have been developed in the literature. This study will compare several of them, which we will now introduce in detail. 


%Homomorphic cryptosystems
%Some examples of homomorphic cryptosystems include the Goldwasser--Micali cryptosystem~\cite{goldwasser_probabilistic_1984} and the Paillier cryptosystem~\cite{stern_public-key_1999}. Even some classic public-key cryptosystems are homomorphic, both RSA and ElGamal are multiplicatively homomorphic, while elliptic curve cryptosystems are additively homomorphic~\cite{li_elliptic_2012}. These cryptosystems, while relatively efficient, only support a limited set of operations on the encrypted data.

%However, there exists \textit{fully homomorphic} cryptosystems, which are not limited to a fixed set of operations, but allow any arbitrary operations on the ciphertext. The first  fully homomorphic cryptosystem was presented by Gentry in 2009. Gentry's cryptosystem, which uses lattice-based cryptography, allows the computation of computation of arbitrary Boolean circuits on binary data~\cite{gentry_fully_2009, shortell_secure_2016}. New algorithms based on Gentry's initial ideas have since been made~\cite{ sen_homomorphic_2013}. One notable example is fully homomorphic cryptosystem by Smart and Vercauteren,~\cite{hutchison_fully_2010}, later improved by Gentry and Halevi ~\cite{hutchison_implementing_2011} which relies on cyclotomic number fields. In 2016, Dasgupta and Pal proposed a fully homomorphic cryptosystem based on polynomial rings~\cite{dasgupta_design_2016}. These two cryptosystems are examples of fully homomorphic cryptosystems which have significantly simpler encryption, decryption, addition and multiplication algorithms.
