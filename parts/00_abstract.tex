\begin{thesisabstract}
    \noindent
    Homomorphic cryptography allows for encrypted data to be modified and operated on without requiring decryption, although current homomorphic cryptosystems are either limited in their permitted operations or are significantly more time-intensive than standard non-homomorphic cryptosystems. Regardless, homomorphic cryptography has been targeted for use in secure image processing and facial recognition, due to their ability to maintain data privacy. In this paper, we compared the viability of the Paillier, Damg{\aa}rd--Geisler--Kr{\o}igaard (DGK), Dijk--Gentry--Halevi--Vaikuntanathan (DGHV), and Brakerski--Gentry--Vaikuntanathan (BGV) cryptosystems for facial image processing applications, by implementing a software library equipped with these cryptosystems, and comparing their time efficiency and accuracy. Furthermore, as an extension to previous research, we attempt to support non-linear image processing operations by deriving and implementing applicable closed-form approximations. Preliminary results have shown that the Paillier and DGK cryptosystems are comparable in accuracy and may be used for image negation, but only the Paillier cryptosystem is consistent enough to produce reasonable to accurate results.

    % This paper is a sample document that serves as a format and content guideline for undergraduate thesis submissions to the Department of Information Systems and Computer Science. In this section, the abstract, the group should be able to give the readers a clear and concise overview of their study. The section should contain the objectives of the thesis, the methods to be used, and when available, the results of the study, the conclusion, and the recommendations for further work, all based on the intended research objectives. A good abstract should be at most around 150--200 words, or half a page. It should also not contain any references, figures, or equations.
\end{thesisabstract}
