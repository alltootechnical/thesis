\section{Conclusion}
We have shown new closed-form approximations to the logarithm and inverse tangent functions and showed how they can be implemented together with the Paillier cryptosystem to allow for privacy-preserving function calculation. We have compared our approach to other closed-form approximations which require a similar number of secure arithmetic operations to evaluate, and have shown our new approximation procedures can generally yield more accurate results over larger ranges of inputs.

While other approximation procedures for the logarithm and inverse tangent functions may yield better accuracy, the rational approximations shown in this paper are suitable enough for privacy-preserving protocols where the precise input values are not exposed, and arithmetic operations are limited and more computationally expensive.

\section{Recommendations}
Further research can be done regarding implementation of other transcendental functions, either by using known infinite series, or by using identities with the quadrature-based approximations we have presented in this paper. Different methods for numerical integration can also be explored and applied for use in deriving formulas for privacy-preserving computation. Furthermore, testing with other types of homomorphic cryptosystems can be done, to potentially reduce noise encountered in secure floating-point operations.
