\section{Experiments on Accuracy}
We implemented privacy-preserving floating-point arithmetic using the system described in Section~\ref{sec:fp_arithmetic}, through the \texttt{python-paillier} library (\url{https://github.com/n1analytics/python-paillier}) running on Python~3.6.5.

We first summarize the approximations discussed in Sections~\ref{sec:logarithm_approximation} and~\ref{sec:arctan_approximation} in Table~\ref{tab:approximation_summary}.

\begin{table*}[ht]
	\caption{Summary of approximations for logarithm and inverse tangent}
	\label{tab:approximation_summary}
    \begin{tabular}{
        p{\dimexpr 0.4\linewidth-2\tabcolsep}
        p{\dimexpr 0.6\linewidth-2\tabcolsep}}
		\toprule
		Approximation & Derivation and formula\\
		\midrule
            Original quadrature for $\log\left(1+x\right)$ (equation \ref{eq:standard_logarithm_quadrature})
            & Let $x=1/n$. Perform five-point Gauss--Legendre quadrature to approximate $\int_{n}^{n+1}{\frac{1}{t}\diff t} = \log\left(1+x\right)$ This yields the approximation:
            \begin{center}
                $\displaystyle 
            \frac{137x^5 + 2310x^4 + 9870x^3 + 15120x^2 + 7560x}
            {30x^5 + 900x^4 + 6300x^3 + 16800x^2 + 18900x + 7560}$.
            \end{center}
            \\[10pt]

            Scaled quadrature for $\log\left(1+x\right)$ (equation \ref{eq:optimal_log_approximation})
            & Let $x=1/n$ and $\alpha= 1/16$. Perform five-point Gauss--Legendre quadrature to approximate $\int_{n}^{\alpha n + \alpha}{\frac{1}{t}\diff t} = \log\left(1+x\right) + \log\alpha$. 
            Then subtract $\log\alpha$, which can be computed in the plaintext domain. This yields the approximation:
            \begin{center}
            $\displaystyle 
            \frac{137x^5 + 26685x^4 + 617370x^3 - 6498630x^2 - 121239315x - 257804775}
            {30(x^5 + 405x^4 + 27210x^3 + 488810x^2 + 2536005x + 3122577)}
            + \log{16}$.
            \end{center}
            \\[10pt]
            
            Partial sum of Euler's series for $\arctan{x}$ (equation \ref{eq:arctan_euler_partial})
            & Compute the partial sum consisting of first five terms of Euler's series for the inverse tangent:
            $\sum_{n=0}^{4}
            {
            \frac
                {2^{2n}(n!)^2}
                {(2n+1)!}
            \frac
                {x^{2n+1}}
                {(1+x^2)^{n+1}}
            }$. This yields the approximation:
            \begin{center}
                $\displaystyle 
                \frac
                {965x^9 + 2370x^7 + 2688x^5 + 1470x^3 + 315x}
                {315\left(x^{10} + 5x^8 + 10x^6 + 10x^4 + 5x^2 + 1\right)}$.
            \end{center}
            \\[10pt]
            
            Quadrature approximation for $\arctan{x}$ (equation \ref{eq:arctan_quadrature})
            & Perform five-point Gauss--Legendre quadrature to approximate $\int_0^x{\frac{1}{1+t^2}\diff t} = \arctan{x}$. This yields the approximation:
            \begin{center}
                $\displaystyle
            \frac
            {4\left(225x^9 + 15925x^7 + 144753x^5 + 350595x^3 + 238140x\right)}
            {15\left(x^{10} + 480x^8 + 11760x^6 + 64120x^4 + 114660x^2 + 63504\right)}$.
            \end{center}
            \\
	    \bottomrule
    \end{tabular}
\end{table*}