\documentclass[sigconf,balance]{acmart}
\usepackage[utf8]{inputenc}
\usepackage{multicol}
\usepackage{multirow}
\usepackage{amsmath}
\usepackage{amssymb}
\usepackage{xinttools}
\usepackage{array}
\usepackage{booktabs} % For formal tables
\usepackage{pgf, tikz}
\usepackage{pgfplots}

% for backwards compatibility
\usepackage{graphicx}

\usetikzlibrary{calc}

\setcopyright{rightsretained}
\newtheorem{theorem}{Theorem}

% DOI
\acmDOI{}

% ISBN
\acmISBN{}

% Conference
\acmConference[PCSC2019]{Philippine Computing Science Congress}{March 2019}{City of Manila, Philippines}
\acmYear{2019}

\acmArticle{}
\acmPrice{}

% These commands are optional
%\acmBooktitle{Transactions of the ACM Woodstock conference}
% \editor{}
% \editor{}
% \editor{}


\begin{document}
\title{Privacy-Preserving Approximation of Transcendental Functions}

\author{Aldrich Ellis C. Asuncion}
\affiliation{%
  % \department{Department of Information Systems and Computer Science}
  \institution{Ateneo de Manila University}
  \streetaddress{Katipunan Avenue, Loyola Heights}
  \city{Quezon City}
  \country{Philippines}
  \postcode{1108}
}
\email{aldrich.asuncion@obf.ateneo.edu}

\author{Brian Christopher T. Guadalupe}
\affiliation{%
  % \department{Department of Information Systems and Computer Science}
  \institution{Ateneo de Manila University}
  \streetaddress{Katipunan Avenue, Loyola Heights}
  \city{Quezon City}
  \country{Philippines}
  \postcode{1108}
}
\email{brian.guadalupe@obf.ateneo.edu}

\author{William Emmanuel S. Yu}
\affiliation{%
  % \department{Department of Information Systems and Computer Science}
  \institution{Ateneo de Manila University}
  \streetaddress{Katipunan Avenue, Loyola Heights}
  \city{Quezon City}
  \country{Philippines}
  \postcode{1108}
}
\email{wyu@ateneo.edu}

% \maketitle
% The default list of authors is too long for headers.
\renewcommand{\shortauthors}{A. Asuncion, B. Guadalupe, and W. Yu}

\begin{abstract}
\begin{thesisabstract}
    \noindent
    Homomorphic cryptography allows for encrypted data to be modified and operated on without requiring decryption, although current homomorphic cryptosystems are either limited in their permitted operations or are significantly more time-intensive than standard non-homomorphic cryptosystems. Regardless, homomorphic cryptography has been targeted for use in secure image processing and facial recognition, due to their ability to maintain data privacy. We compared the viability of the Paillier, Damg{\aa}rd--Geisler--Kr{\o}igaard (DGK), Dijk--Gentry--Halevi--Vaikuntanathan (DGHV), and Brakerski--Gentry--Vaikuntanathan (BGV) cryptosystems for facial image processing applications by implementing a software library equipped with these cryptosystems and comparing their time efficiency and accuracy. We attempt to support non-linear image processing operations by deriving and implementing applicable closed-form approximations. Results have shown that the Paillier and DGK cryptosystems are comparable in accuracy and may be used for image negation, but only the Paillier cryptosystem is consistent enough to produce reasonable to accurate results for logarithm intensity transformation. The DGHV and BGV cryptosystems were found to be significantly slower and not applicable for practical use.

    % This paper is a sample document that serves as a format and content guideline for undergraduate thesis submissions to the Department of Information Systems and Computer Science. In this section, the abstract, the group should be able to give the readers a clear and concise overview of their study. The section should contain the objectives of the thesis, the methods to be used, and when available, the results of the study, the conclusion, and the recommendations for further work, all based on the intended research objectives. A good abstract should be at most around 150--200 words, or half a page. It should also not contain any references, figures, or equations.
\end{thesisabstract}

\end{abstract}

%
% The code below should be generated by the tool at
% http://dl.acm.org/ccs.cfm
% Please copy and paste the code instead of the example below.
%
\begin{CCSXML}
    <ccs2012>
    <concept>
    <concept_id>10002950.10003714.10003736.10003737</concept_id>
    <concept_desc>Mathematics of computing~Approximation</concept_desc>
    <concept_significance>500</concept_significance>
    </concept>
    <concept>
    <concept_id>10002950.10003714.10003740</concept_id>
    <concept_desc>Mathematics of computing~Quadrature</concept_desc>
    <concept_significance>500</concept_significance>
    </concept>
    <concept>
    <concept_id>10002978.10002991.10002995</concept_id>
    <concept_desc>Security and privacy~Privacy-preserving protocols</concept_desc>
    <concept_significance>500</concept_significance>
    </concept>
    <concept>
    <concept_id>10002978.10002979.10002981.10011745</concept_id>
    <concept_desc>Security and privacy~Public key encryption</concept_desc>
    <concept_significance>300</concept_significance>
    </concept>
    </ccs2012>
\end{CCSXML}

\ccsdesc[500]{Mathematics of computing~Approximation}
\ccsdesc[500]{Mathematics of computing~Quadrature}
\ccsdesc[500]{Security and privacy~Privacy-preserving protocols}
\ccsdesc[300]{Security and privacy~Public key encryption}

\keywords{numerical methods, Gauss--Legendre quadrature, floating-point arithmetic, transcendental functions, homomorphic encryption, secure computation, Paillier cryptosystem}
% \keywords{cryptography; homomorphic encryption; data privacy; image processing; intensity transformation; facial recognition; numerical approximations}

\maketitle

\section{Introduction}
Digital image security is a growing concern, and various image encryption schemes have been developed and analyzed in the literature~\cite{murugan_survey_2018, jain_image_2016, khoirom_cryptanalysis_2018}. Cryptosystems allow for the secure transmission of data by way of encryption and decryption schemes. 
Of particular interest is homomorphic image encryption, which allows operations to be performed on encrypted images. These operations are then preserved when the images is decrypted. In this manner, images are manipulated without compromising data privacy, as the image being operated on remains secure~\cite{fontaine_survey_2007, sen_homomorphic_2013}. 

In the Philippine setting, preservation of data privacy is of utmost importance when it comes to outsourcing work involving sensitive data to local companies with the ratification of the Data Privacy Act of 2012. Thus recent developments in applications of homomorphic encryption on non-text-based data (such as images) are becoming relevant today.

In the area of homomorphic image encryption, there is interest in being able to perform image manipulation operations such as image adjustment, image filtering, and morphological extraction operations~\cite{ziad_cryptoimg:_2016, gonzalez_digital_2008}. There is also work on facial recognition on encrypted images~\cite{turk_eigenfaces_1991, hutchison_privacy-preserving_2009}. While secure linear image operations have been implemented~\cite{ziad_cryptoimg:_2016}, preprocessing which requires non-linear operations, such as those in~\cite{oravec_illumination_2010}, has yet to be performed in the encrypted domain.

Homomorphic encryption schemes have two main limitations: first, they are limited in the operations that can be performed on encrypted data efficiently~\cite{li_elliptic_2012}, and second, they are slower than non-homomorphic schemes~\cite{sen_homomorphic_2013}. This paper presents implementations of non-linear image operations under a homomorphic encryption scheme, and evaluates their applicability and performance.

To investigate the applicability of homomorphic image manipulation, in Section \ref{sec:chapter_2}, we introduce three homomorphic cryptosystems prominent in the literature: the Paillier, Damg{\aa}rd--Geisler--Kr{\o}igaard (DGK), and Brakerski--Gentry--Vaikuntanathan (BGV) cryptosystems, as well as relevant past implementations. In Section \ref{sec:chapter_3}, we implement a software library allowing for common linear and non-linear image processing operations using each of the three cryptosystems, assuming a client-server model for secure computation.

To evaluate the performance of homomorphic image manipulation, in Section \ref{sec:chapter_4}, we test the time efficiency and accuracy of common image processing operations under homomorphic encryption. We test for image quality of the resulting images using mean squared error (MSE), peak signal to noise ratio (PSNR), and structural similarity index (SSIM), which are measures of image quality used for assessing image encryption schemes~\cite{ahmed_benchmark_2016}.

% Training and test data are taken from the publicly available \texttt{faces94} dataset (\url{https://cswww.essex.ac.uk/mv/allfaces/faces94.html}) maintained by the University of Essex.

% \subsection{Significance of the Study}
% This study aims to directly address one of the current problems in the research of homomorphic cryptosystems: the practicality of homomorphic encryption~\cite{sen_homomorphic_2013}. The literature shows existing homomorphic cryptosystems and how they support primitive operations (usually addition and multiplication) on encrypted data, however, for homomorphic cryptosystems to be useful for practical use, efficient support for more complicated image processing operation must be demonstrated as well. While implementations of image processing operations in a homomorphic cryptosystem exist in the literature~\cite{ziad_cryptoimg:_2016, garay_algorithms_2014}, our study targets the more computationally intensive task of facial recognition and detection.



% We wish to contribute to existing models for the secure transmission and modification of image data by developing a software library which allows the use and comparison of various homomorphic encryption algorithms. Our software library would allow service providers to perform image manipulation for clients without compromising data privacy.

\section{Review of Related Literature}
\label{sec:chapter_2}
\subsection{Homomorphic Cryptosystems}
A cryptosystem~\cite{bauer_cryptosystem_2005} consists of an encryption function and a decryption function, which operate on plaintexts, ciphertexts and keys. A \textit{plaintext} is text that can be commonly understood within a larger group. Given a \textit{key}, the encryption function maps a plaintext to some \textit{ciphertext}, which can only be understood by authorized parties. The decryption function similarly maps a ciphertext back to its corresponding plaintext, given an appropiate key. By sending data as ciphertext and only exposing the corresponding plaintexts to authorized parties who have access to the appropiate keys, secure data transmission can be achieved.

A \textit{homomorphic cryptosystem} allows operations to be performed on recovered plaintexts by performing corresponding operations on ciphertexts. This allows for secure computation by performing operations in the ciphertext domain. It is important to note that a simple operation in the plaintext space may require a computationally intensive operation in the ciphertext space.

% big table to summarize Paillier and DGK
% \begin{table*}[ht]
% 	\caption{Summary of partially homomorphic cryptosystems}
% 	\label{tab:phe_summary}
%     \begin{tabular}{
%         p{\dimexpr 0.2\linewidth-2\tabcolsep}
%         p{\dimexpr 0.4\linewidth-2\tabcolsep}
%         p{\dimexpr 0.4\linewidth-2\tabcolsep}}
% 		\toprule
% 		 & Paillier & DGK\\
%         \midrule
%             Plaintext space &
%             $\mathbb{Z}_n$ &
%             $\mathbb{Z}_u$ 
%             \\
%             Ciphertext space &
%             $\mathbb{Z}_{n^2}$ &
%             $\mathbb{Z}^*_n$
%             \\
%             Key generation &
%             Define a function $L(x)$ as the largest integer $v$ greater than zero such that $x-1 \geq vn$.\newline
%             Choose two large primes $p$ and $q$, and set $n = pq, \lambda = \mathrm{lcm}(p-1,q-1)$.\newline
%             Select an integer $g$, $0\leq g \leq n^2$ such that $\mathrm{gcd}(L(g^\lambda \bmod n^2), n) = 1$\newline
%             The public key as $(g,n)$ and the private key $(p,q)$
%             &
%             Let $p,q$ be primes such that we can choose two primes $v_p$ and $v_q$ such that $v_p | (p-1)$ and $v_q | (q-1)$, and a small prime $u$ such that $u | (p-1)$ and $u | (q-1)$. \newline
%             Denote $n = pq$. \newline
%             Choose $g$ to be an integer of order $uv_pv_q$ and $h$ to be of order $v_pv_q$.\newline
%             The public key is $(n,g,h,u)$ and the private key is $(p,q,v_p,v_q)$.
%             \\
%             Encryption function &
%             $E(m) = g^m \cdot r^n \mod{n^2}$ &
%             $E(m) = g^m \cdot h^r \mod{n}$
%             \\
%             Decryption function &
%             $D(c) = L(c^\lambda \bmod n^2) \times (L(g^\lambda \bmod n^2))^{-1} \mod n$ &
%             Compute $c^{v_pv_q} \bmod n$. There is a one-to-one correspondence between this quantity and the plaintexts.
%             \\
% 	    \bottomrule
%     \end{tabular}
% \end{table*}
\subsection{The Paillier and DGK Cryptosystems}
The Paillier cryptosystem \cite{stern_public-key_1999}, developed by Pascal Paillier, is a probabilistic encryption scheme which is based on the composite residuosity class problem.
Table \ref{tab:paillier_summary} shows the key generation, encryption, and decryption procedures of the Paillier cryptosystem.
\begin{table}[ht]
	\caption{The Paillier cryptosystem}
	\label{tab:paillier_summary}
    \begin{tabular}{
        p{\dimexpr 0.20\linewidth-2\tabcolsep}
        p{\dimexpr 0.80\linewidth-2\tabcolsep}}
		\toprule
		Step & Description\\
        \midrule
            Generate\newline Keys &
            Let $L(x)$ be the largest integer $v$ greater than zero such that $x-1 \geq vn$.\newline
            Choose two large primes $p$ and $q$, and set $n = pq$, and $ \lambda = \mathrm{lcm}(p-1,q-1)$.\newline
            Select an integer $g$, $0\leq g \leq n^2$ such that $\mathrm{gcd}(L(g^\lambda \bmod n^2), n) = 1$\newline
            The public key as $(g,n)$ and the private key is $(p,q)$.
            \\
            Encrypt &
            The encryption of a plaintext $m \in  \mathbb{Z}_n$ given the public key is  $E(m) = g^m \cdot r^n \mod{n^2}$, where $r$ is a random integer in $\mathbb{Z}_{n^2}$.
            \\
            Decrypt &
            The decryption of a plaintext $c \in  \mathbb{Z}_{n^2}$ given the private key is $D(c) = L(c^\lambda \bmod n^2) \times (L(g^\lambda \bmod n^2))^{-1} \mod n$. 
            \\
	    \bottomrule
    \end{tabular}
\end{table}
% The scheme allows for the encryption and decryption of integer messages, and is known to be additively homomorphic.
% We now state the encryption and decryption algorithms of the Paillier cryptosystem and its homomorphic properties.

% \paragraph{Key Generation}
% We first define a function $L(x)$ as the largest integer $v$ greater than zero such that $x-1 \geq vn$.
% We choose two large primes $p$ and $q$, and set $n = pq, \lambda = \mathrm{lcm}(p-1,q-1)$.
% Then we select an integer $g$, $0\leq g \leq n^2$ such that $\mathrm{gcd}(L(g^\lambda \bmod n^2), n) = 1$.
% We denote the public key as $(g,n)$ and the private key $(p,q)$.

% \paragraph{Encryption and Decryption}
% The encryption function to encrypt a plaintext $m \in \mathbb{Z}_n$ given a public key $(g,n)$ is defined as
% \begin{align*}
%   E(m) = g^m \cdot r^n \mod{n^2},
% \end{align*}
% where $r$ is a random non-negative integer less than $n^2$.

% The decryption function to decrypt a ciphertext $c \in \mathbb{Z}^\ast_{n^2}$ given a private key $(p,q)$ is defined as:
% \begin{align*}
%   D(c) = L(c^\lambda \bmod n^2) \times (L(g^\lambda \bmod n^2))^{-1} \mod n
% \end{align*}

For all plaintexts $m_1,m_2 \in \mathbb{Z}_n$, the following properties hold for the Paillier cryptosystem:
\begin{align}
  D(E(m_1)g^{m_2}\bmod n^2) &=(m_1+m_2)\bmod n \label{eq:paillier_ctpluspt} \\ 
  D(E(m_1)E(m_2)\bmod n^2) &=(m_1+m_2)\bmod n \label{eq:paillier_ctplusct} \\ 
  D(E(m_1)^{m_2}\bmod n^2) &= m_1m_2\bmod n. \label{eq:paillier_cttimespt}
\end{align}
These identities allow for the addition of a ciphertext with a plaintext (Equation \ref{eq:paillier_ctpluspt}), the addition of two ciphertexts (Equation \ref{eq:paillier_ctplusct}), and the multiplication of a ciphertext by a plaintext (Equation \ref{eq:paillier_cttimespt}) under the Paillier cryptosystem.

The DGK cryptosystem was published by Damg{\aa}rd, Geisler, and Kr{\o}igaard in 2007 in an effort to create a secure integer comparison scheme \cite{pieprzyk_efficient_2007, cryptoeprint:2008:321} which is widely used in the literature \cite{veugen_improving_2012}.
Table \ref{tab:dgk_summary} shows the key generation, encryption, and decryption procedures of the DGK cryptosystem. 

Similar to the Paillier cryptosystem, for all $m_1,m_2 \in \mathbb{Z}_u$, the following homomorphic properties hold for the DGK cryptosystem:
\begin{align}
    D(E(m_1)g^{m_2}) &=(m_1+m_2)\bmod u \label{eq:dgk_ctpluspt}\\
    D(E(m_1)E(m_2)) &=(m_1+m_2)\bmod u \label{eq:dgk_ctplusct}\\
    D(E(m_1)^{m_2}) &= m_1m_2\bmod u. \label{eq:dgk_cttimespt},
\end{align}
which allow for homomorphic operations similar to to that of the Paillier cryptosystem.
As the multiplicative homomorphism (Equation \ref{eq:dgk_cttimespt}) was not presented in the original paper, we provide a short proof here.
\begin{table}[ht]
	\caption{The DGK cryptosystem}
	\label{tab:dgk_summary}
    \begin{tabular}{
        p{\dimexpr 0.2\linewidth-2\tabcolsep}
        p{\dimexpr 0.8\linewidth-2\tabcolsep}}
		\toprule
		Step & Description\\
        \midrule
            Generate \newline Keys &
            Let $p,q$ be primes such that we can choose two primes $v_p$ and $v_q$ such that $v_p | (p-1)$ and $v_q | (q-1)$, and a small prime $u$ such that $u | (p-1)$ and $u | (q-1)$. \newline
            Denote $n = pq$. \newline
            Choose $g$ to be an integer of order $uv_pv_q$ and $h$ to be of order $v_pv_q$.\newline
            The public key is $(n,g,h,u)$ and the private key is $(p,q,v_p,v_q)$.
            \\
            Encrypt &
            The encryption of a plaintext $m \in \mathbb{Z}_u$ given the public key is $E(m) = g^m \cdot h^r \mod{n}$, where $r$ is a random integer in $\mathbb{Z}_n$.
            \\
            Decrypt &
            To decrypt a ciphertext $c \in \mathbb{Z}_n^\ast$, we compute $c^{v_pv_q} \bmod n$. There is a one-to-one correspondence between this quantity and the plaintexts, so the plaintext can be recovered using a lookup table.
            \\
	    \bottomrule
    \end{tabular}
\end{table}


\begin{theorem}
    In the DGK cryptosystem with public key $(n,g,h,u)$ and private key $(p,q,v_p,v_q)$, it holds that $D(E(m_1)^{m_2}) = m_1m_2\bmod u$ for all plaintexts $m_1,m_2 \in \mathbb{Z}_u$.
\end{theorem}
\begin{proof}
  Let $m_1,m_2 \in \mathbb{Z}_u$.
  We note that
  \begin{align*}
    E(m_1)^{m_2} 
    &= (g^{m_1} \cdot h^r \bmod{n})^{m_2} \bmod n \\
    &= (g^{m_1} \cdot h^r)^{m_2} \bmod{n}\\
    &= g^{m_1m_2} \cdot (h^{m_2r}) \bmod{n}.
  \end{align*}
  Since $r$ is a random integer, $m_2r$ is also a random integer, Therefore, $g^{m_1m_2} \cdot (h^{m_2r}) \bmod{n} = E(m_1)^{m_2}$ is a valid encryption of the message $m_1m_2$.
\end{proof}

We have shown how the Paillier and DGK cryptosystems have similar homomorphic properties,allowing the addition of integers and multiplication of ciphertexts by plaintexts.
In order to use these cryptosystems in secure non-linear image processing, we require the two schemes to support the encryption and manipulation of floating-point numbers. To do this, we represent floating-point numbers using an encrypted mantissa and an unencrypted exponent, an approach seen in~\cite{ziad_cryptoimg:_2016}. 

Table \ref{tab:flop_summary} describes two-party protocols which allow a server to perform division~\cite{boukoros_lightweight_2017}, squaring~\cite{hutchison_privacy-preserving_2009}, and multiplication on messages $x$ and $y$ encrypted by a client. These protocols, used in conjunction with the floating-point extension, allow for secure floating-point arithmetic~\cite{pcsc-paper}.

\begin{table}[ht]
	\caption{Two-party FP protocols}
	\label{tab:flop_summary}
    \begin{tabular}{
        p{\dimexpr 0.25\linewidth-2\tabcolsep}
        p{\dimexpr 0.75\linewidth-2\tabcolsep}}
		\toprule
		Operation & Protocol\\
        \midrule
            Division &
            The server selects a random number $r$ and sends $E(rx)$ and $E(ry)$ to the client, who can decrypt to obtain $rx$ and $ry$. The client then computes and encrypts the quotient and sends $E(rx/ry) = E(x/y)$ to the server. 
            \\
            Squaring &
            The server selects a random number $r$ and computes $E\left(x+r\right)$. The server sends $E(x+r)$  to the client, who can decrypt to obtain $x+r$. The client then encrypts and sends $E((x+r)^2)$ to the server. The server then computes $E\left(\left(x+r\right)^2\right)E\left(-2rx + r^2\right) = E\left(x^2\right)$.
            \\
            Multiplication &
            The server obtains $E(x^2)$, $E(y^2)$, and $E((x+y)^2)$ using the exponentiation protocol. It can then compute $\frac{1}{2}E\left(\left(x+y\right)^2\right)E\left(x^2\right)^{-1}E\left(y^2\right)^{-1} = E\left(xy\right)$.\\
	    \bottomrule
    \end{tabular}
\end{table}
\subsection{The BGV Cryptosystem}
The BGV (Brakerski--Gentry--Vaikuntanathan) cryptosystem \cite{cryptoeprint:2011:277} is a fully homomorphic cryptosystem created based on the ring-learning with error problem. A \textit{fully homomorphic cryptosytem} allows aribitrary computation on encrypted data.
However, they are known to be significantly slower than partially homomorphic cryptosytems; improving the efficiency of fully homomorphic cryptosytems is an active area of research \cite{sen_homomorphic_2013}. 
The details regarding the construction of the BGV cryptosystem are beyond the scope of this paper. 

For this study, we will be using \textit{HELib}~\cite{garay_algorithms_2014}, an open-source library which implements the BGV cryptosystem with optimizations. \textit{HELib} supports arithmetic operations on integer ciphertexts and logical operations on binary ciphertexts. 
% Evaluation of operations on 120 inputs in \textit{HELib} was performed in around 4 minutes, with an average of 2 seconds to process a single input~\cite{hutchison_fully_2010,cryptoeprint:2011:566}. 
It has also been adapted to Python using the Pyfhel library \cite{pyfhel_2018} maintained by Ibarrondo, Laurent (SAP) and Onen (EURECOM), and licensed under the GNU GPL v3 license.  

\subsection{Intensity Transformations in CryptoImg}
A paper published in 2009 by Ziad, et al. implements privacy-preserving image processing using \textit{CryptoImg}, a library for the Open Source Computer Vision Library (OpenCV)~\cite{bradski_opencv_2000} which implements various homomorphic encryption and image processing routines using the Paillier homomorphic cryptosystem~\cite{ziad_cryptoimg:_2016}. The \textit{CryptoImg} library assumed a client-server model wherein a client requests image processing operations from a server. A client first encrypts a digital image and sends the image securely to a server, which then operates on the encrypted image without revealing its contents. The resulting image is sent back to the client, which decrypts the image to recover the desired output.
\begin{figure}[!ht]
    \centering
    \includegraphics[width=0.4\textwidth]{figures/ClientServerModel.png}
    \caption{Architecture used by \textit{CryptoImg}, from \cite{ziad_cryptoimg:_2016}}
    \label{fig:clientserver}
\end{figure}

An intensity transformation on a digital image $R$ can be defined as a function $T$ which is applied to every pixel $r$ in $R$. Table \ref{tab:imageoperation_summary} shows common image intensity transformations~\cite{gonzalez_digital_2008}. 
The \textit{CryptoImg} library implemented two types of intensity transformations: image negation and brightness control, which are both linear transformations. 
In this study, we consider two common non-linear transformations. the logarithm transformation and power-law transformation~\cite{gonzalez_digital_2008}, seen in Table \ref{tab:imageoperation_summary}.

% The \textit{CryptoImg} library implemented the following image processing operations: image negation, brightness adjustment, spatial filters (for noise reduction, edge detection and sharpening), morphological operations, and histogram equalization. 


% For image negation and spatial filters, the protocols specified by \textit{CryptoImg} allow all image processing operations to be performed on the server. However, due to limitations in the Paillier cryptosystem, the protocols presented for morphological operations and histogram equalization require both the client and the server to perform image processing calculations, although the server performs a significant portion of the processing.

% Ziad, et al. also showed experimental results establishing the relatively slow performance of image operations under a homomorphic cryptosystem. For instance, while sharpening and applying a Sobel filter each take less than a second when applied to a $512\times 512$ plaintext image, when applied to an encrypted image, sharpening required at least 238.257 seconds, and applying the Sobel filter required at least 147.567 seconds \cite{ziad_cryptoimg:_2016}.


% We now discuss several limitations in the \textit{CryptoImg} study which we focus on for our research. First, the \textit{CryptoImg} library was limited in the image intensity transformations it implemented. We propose additional protocols to support more computationally intensive intensity transformations.
% Second, the \textit{CryptoImg} library only considered the Paillier cryptosystem. We consider testing the performance of other homomorphic cryptosystems, which differ in their processing time and supported operations.

% Convert to single table
\begin{table}[ht]
	\caption{Summary of intensity transformations}
	\label{tab:imageoperation_summary}
    \begin{tabular}{
        p{\dimexpr 0.65\linewidth-2\tabcolsep}
        p{\dimexpr 0.35\linewidth-2\tabcolsep}}
		\toprule
		Operation & Transformation\\
        \midrule
            Image negation &
            $T\left(r\right) = L-1-r$
            \\
            Brightness adjustment transformation with parameter $v$ &
            $T\left(r\right) = r+v$
            \\
            Logarithm transformation with parameter $c$ &
            $T\left(r\right) = c \log\left(1 + r\right)$\\
            Power-law transformation with parameters $c$ and $\gamma$, where $c>0$ and $\gamma > 0$&
            $T\left(r\right) = c r^{\gamma}$\\
	    \bottomrule
    \end{tabular}
\end{table}
% We represent a digital image $R$ as an $M \times N$ matrix of pixel intensity values, each value in the range $\left[0, L-1\right]$, for some positive integer $L$. We denote with $R(x,y)$ the entry at the $x$th row and $y$th column of a matrix $R$.
The logarithm transformation is used to enhance dark pixels or increase the dark details of an image by mapping low intensity values to a wider range of values. A power-law transformation can calibrate the operation of many image capture and output devices such as cameras, printers and displays in a process called \textit{gamma correction}~\cite{gonzalez_digital_2008}. Both of these transformation can be used in intensity normalization \cite{oravec_illumination_2010}, which is used in some facial recognition algorithms to account for differences in lighting which make facial recognition difficult.

To implement non-linear intensity transformations in a privacy--preserving system, it is necessary to evaluate logarithm and exponential functions using closed--form approximations, since the arguments of the functions are unknown. An applicable approximation for the logarithm, found in \cite{pcsc-paper}, is
% TODO: HOW ABOUT BGV
\begin{align}\label{eq:optimal_log_approximation}
	\begin{split}
		&\log\left(1+x\right) \approx \frac{a(x)}{b(x)} + \log{16} ,
	\end{split}
\end{align}
where
\begin{align*}
	\begin{split}
	a(x) &= 137x^5 + 26685x^4 + 617370x^3 - 6498630x^2 \\
	&- 121239315x - 257804775\\
	b(x) &= 30(x^5 + 405x^4 + 27210x^3 + 488810x^2 + 2536005x \\
	&+ 3122577).
	\end{split}
\end{align*}
This approximation can be computed under a homomorphic cryptosystem which supports floating-point operations by precomputing the value of $\log{16}$ in the plaintext domain.
% \begin{align}\label{eq:optimal_log_approximation}
% 	\begin{split}
% 		&\log\left(1+x\right) \\
% 		&\approx \frac{137x^5 + 26685x^4 + 617370x^3 - 6498630x^2 - 121239315x - 257804775}
% 		{30(x^5 + 405x^4 + 27210x^3 + 488810x^2 + 2536005x + 3122577)}\\
% 		&+ \log{16}.
% 	\end{split}
% \end{align}

% \subsubsection{Power-Law Transformation}
To apply the power-law transformation we can use partial sums of the infinite series,
\begin{align} \label{eq:power_approximation}
	x^\gamma &= \sum_{n=0}^{\infty}{\frac{(\gamma\log{x})^n}{n!}},
\end{align}
which relies on the logarithm function.
For the implementation of the power-law transformation for this paper, a partial sum consisting of the first five terms of the infinite series were used.

\subsection{Facial Detection and Recognition}

Since simple image operations mentioned earlier can be done within a homomorphic cryptosystem, these operations can be assembled together in order to do more complex operations. A prominent application of image processing that often uses complex image operations is \textit{facial recognition}.

Traditional facial recognition algorithms rely on detecting salient features in a face image. One of the popular facial recognition algorithms is \textit{eigenfaces}.
% Traditional facial recognition algorithms rely on detecting salient features in a face image. Two of the popular facial recognition algorithms are \textit{eigenfaces} and \textit{Haar cascades}.

\subsubsection{Eigenfaces}
Proposed by Matthew Turk and Alex Pentland, the eigenfaces method uses principal component analysis (PCA) in order to express an input image as a linear combination of eigenfaces \cite{turk_eigenfaces_1991}. An \textit{eigenface} is a principal component, or more simply an eigenvector that represents a certain variation between the face images which are taken from the initial training set. The number of resulting eigenfaces is equal to the number of face images in the training set.

% In the enrollment process, $M$ face images $\Theta_1, \ldots, \Theta_M$, each of size $p \times q$, are taken in to comprise the initial training set. The training images can be represented as vectors of length $N = pq$, where each row of an image is concatenated together.

% The average of the training images, denoted as $\Psi$, is
% \begin{equation}
% 	\Psi = \frac{1}{M} \sum_{i=1}^{M} \Theta_i
% \end{equation}

% The \textit{difference vectors} are then computed as $\Phi_i = \Theta_i - \Psi$. PCA is applied on the covariance matrix of the vectors
% \begin{equation}
% 	\mathbf{C} = \frac{1}{M} \sum_{i=1}^M \Phi_i \Phi_i^\top = \frac{1}{M} \mathbf{A}\mathbf{A}^\top,
% \end{equation}
% where $\mathbf{A}$ is an $N \times M$ matrix defined by $\left[\Theta_1 \quad \Theta_2 \quad \cdots \quad \Theta_M\right]$, in order to determine the orthonormal eigenvectors \cite{hutchison_privacy-preserving_2009}.

% Directly computing the covariance matrix and then applying PCA will become inefficient for large sizes of $\mathbf{A}$, since computing for $\mathbf{A}\mathbf{A}^\top$ results in an $N \times N$ matrix, which can get drastically large because it is dependent on the size of the image. On the other hand, computing for $\mathbf{A}^\top\mathbf{A}$ only results in an $M \times M$ matrix, which is much smaller than the previous one because it is just dependent on the number of face images in the training set. Now, we can apply PCA to $\mathbf{A}^\top\mathbf{A}$ along with some post-processing to obtain the eigenvectors \cite{hutchison_privacy-preserving_2009}.

% Instead of getting the eigendecomposition of the covariance matrix by explicitly computing the eigenvalues and eigenvectors of $\mathbf{C}$, we can also apply \textit{singular value decomposition} (SVD) on the matrix $\Phi^\top$. This is because SVD works through a divide-and-conquer method which results in greater numerical stability, while eigendecomposition simply uses the traditional QR factorization \cite{nakatsukasa_stable_2013, gu_divide-and-conquer_1995}. Upon applying SVD, we now get the eigenvectors $\mathbf{u}_1, \ldots, \mathbf{u}_M$ and their corresponding eigenvalues $\lambda_1, \ldots, \lambda_M$.

% We choose $K$ eigenvectors $\mathbf{u}_1, \ldots, \mathbf{u}_K$ such that it comprises a set associated with the $K$ largest eigenvalues, where $K$ is much smaller than $M$. This set is now the \textit{face space}. Then, the images $\Theta_1, \ldots, \Theta_K$ are projected onto the face space spanned by the eigenfaces to determine the weight vectors $\Omega_1, \ldots, \Omega_K$.

% To perform the recognition process, the algorithm projects the input image $\Gamma$ onto the face space, and then comparing the projected image $\bar{\Omega} = \mathbf{u}_i\left(\Gamma - \Psi\right)$ with each eigenface in the face space using a metric such as the Euclidean distance. Thus it is computed as $d_i = \left\lVert \Omega_i -\bar{\Omega} \right\rVert$ for all $i=1,\ldots,K$, where $\left\lVert \cdot \right\rVert$ denotes the Euclidean norm.

% Then, a match can be reported if the smallest possible distance is smaller than the given threshold value $T$.
% \newcommand{\argmin}{\mathop{\mathrm{arg\,min}}}
% \begin{equation}
% 	\text{ID} = \begin{cases}\argmin_i d_i & \text{if } \min_i d_i \le T \\\emptyset & \text {otherwise}\end{cases}
% \end{equation}


\subsection{Previous Implementations of Secure Facial Recognition}
Erkin, et al. \cite{hutchison_privacy-preserving_2009} devised a method to incorporate the use of homomorphic cryptosystems into the eigenfaces method. In their study, they proposed a two-party system where Alice holds an encrypted image $\left[\Gamma\right]$, while Bob maintains a database of $K$ eigenfaces $\mathbf{u}_1, \ldots, \mathbf{u}_K$, and feature vectors $\Omega_1, \ldots, \Omega_M$ in the clear.

\begin{figure}[h]
    \centering
    \includegraphics[width=0.4\textwidth]{figures/secure_eigenfaces.png}
    \caption{Diagram of the privacy-preserving facial recognition process \cite{hutchison_privacy-preserving_2009}}
\end{figure}

In order to ensure privacy, the steps in the eigenfaces algorithm, namely: projection, distance computation, and match finding, are done within the encrypted domain, i.e., using the operations in the Paillier cryptosystem.

Projection is similar to that of the original eigenfaces algorithm, except the operations are replaced with their respective operations in the cryptosystem. Distance computation in this version is somewhat different from the original eigenfaces method, in that it deals with the square of the Euclidean distance since the relative order of the distances is only important when comparing these during the match finding step \cite{hutchison_privacy-preserving_2009}.
\begin{align}
    d_i &= \left\lVert \Omega_i - \bar{\Omega} \right\rVert ^2 = \sum_{j=1}^{K} \left(\omega_{ij} - \bar{\omega}_j\right)^2 \\
        &= \underbrace{\sum_{j=1}^{K} \omega_{ij}^2}_{\mathcal{S}_1} + \underbrace{\sum_{j=1}^{K} \left(-2 \omega_{ij} \bar{\omega_j}\right)}_{\mathcal{S}_2} + \underbrace{\sum_{j=1}^{K} \bar{\omega}_{j}^2}_{\mathcal{S}_3}
\end{align}

Computing for the distances within Paillier would just be multiplying the encrypted sums together.
\begin{equation}
	\left[d_i\right] = \left[\mathcal{S}_1\right] \cdot \left[\mathcal{S}_2\right] \cdot \left[\mathcal{S}_3\right]
\end{equation}

The terms $\left[\mathcal{S}_1\right]$ and $\left[\mathcal{S}_2\right]$ can be easily computed by Bob, since he already knows both $\omega_i$ in the clear and $\left[\bar{\omega}_i\right]$ which is in encrypted form. Computing for $\left[\mathcal{S}_3\right]$ is trickier because Bob cannot compute for $\left[\bar{\omega}_i^2\right]$ because pairwise multiplication is not supported in Paillier, that is why Bob needs help from Alice to square a number through a protocol described below \cite{hutchison_privacy-preserving_2009}.

Before Bob sends $\left[\bar{\omega}_i\right]$ to Alice for squaring, he adds a random number $r_i$ to compute $\left[x_i\right] = \left[\bar{\omega}_i + r_i\right] = \left[\bar{\omega}_i\right] \cdot \left[r_i\right]$, where $r_i$ is obviously distinct for every $i$, then sends $\left[x_i\right]$ to Alice. She then decrypts it and computes $x_i^2$, and then computes $\mathcal{S}_3^\prime = \sum_{i=1}^{K} x_i^2$, after which she encrypts the sum and sends $\left[\mathcal{S}_3^\prime\right]$ to Bob. Now, he can compute for $\left[\mathcal{S}_3\right]$ as follows:
\begin{equation}
	\left[\mathcal{S}_3\right] = \left[\mathcal{S}_3^\prime\right] \cdot \prod_{j=1}^{K} \left(\left[\bar{\omega}_j\right]^{-2r_j} \cdot \left[-r_j^2\right]\right)
\end{equation}

The protocol for squaring a number as described earlier is a workaround for the limitations of Paillier or any other partially homomorphic cryptosystem that only supports addition and scalar multiplication.
% This can be possibly extended so that operations such as pairwise multiplication and exponentiation are also supported, provided that the protocol involves two parties.

The match-finding step is done by comparing the distances obtained from the previous step to a specified threshold $T$. If the minimum distance is smaller than $T$, then a match is found, and the encrypted identity of the match is returned to Alice \cite{hutchison_privacy-preserving_2009}. Getting the minimum among the encrypted distances would involve comparing two encrypted numbers, which Paillier does not support. Instead, the DGK cryptosystem \cite{pieprzyk_efficient_2007} is used for the comparison protocol.

\chapter{METHODOLOGY}

The study consists of two parts: first, the implementation of the three homomorphic encryption schemes and image processing operations under each scheme as an OpenCV library. The second part of the study consists of assessing the robustness and security of image operations under each encryption scheme using benchmarks in \cite{ahmed_benchmark_2016}.

\section{Implementation of the OpenCV library}

The current version of the OpenCV library (3.4.1) will be forked from the open-source GitHub repository at https://github.com/opencv \cite{opencv_library}.

Three homomorphic encryption schemes will be implemented, those presented by Ziad, et al. \cite{ziad_cryptoimg:_2016}, Li, et al. \cite{li_elliptic_2012} and Smart and Vercauteren \cite{hutchison_fully_2010}. The encryption and decryption methods will be implemented in C++.

Aside from implementing the homomorphic encryption and decryption algorithms, we will also implement library functions for the following image processing functions, as they are defined in \cite{gonzalez_digital_2008}. We will also take note of whether or not a candidate operation is impossible to perform under a given cryptosystem.  We let $R(x,y)$ denote the intensity at coordinate $(x,y)$ in the source image, and $S(x,y)$ denote the intensity at coordinate $(x,y)$ in the resulting image. We further suppose that the intensity values of pixels are in the range $[0, L-1]$.
\begin{description}
	\item[Intensity transformations.] Transformations on the intensities of each of the pixels on an image. The following definitions hold for all $x,y$.
	\begin{enumerate}
		\item Image negation: $S(x,y) = L - 1 - R(x,y)$.
		\item Log transformation: $S(x,y) = c\log{(1 + R(x,y))}$, $c \geq 0$.
		\item Power-law transformation: $S(x,y) = c[R(x,y)]^\gamma$, $c > 0, \gamma > 0$.
	\end{enumerate}
	\item[Spatial filters.] Filters implemented by performing a convolution between an $M\times N$ source image and an $m\times n$ filter matrix. Let $W$ be a filter matrix. Then the corresponding spatial filter is given by
	\begin{align}
		S(x,y) = \sum_{s=1}^m{\sum_{t=1}^n{W(s,t)R(x+s,y+t)}}.
	\end{align}
	\item [Histogram processing / equalization]

\end{description}

\section{Assessment of each homomorphic encryption scheme}

We will first obtain standard test images from \cite{gonzalez_image_nodate}. Both greyscale and color images will be used.

For each plaintext image (PT), we will consider each image operation listed above and generate three images: a plaintext domain transformation (PDT), a ciphertext image (CT), and an encrypted domain transformation (EDT). The cipherThe PDT will be generated by running the image operation on the original image. The CT will be generated by encrypting the image, then applying the operation, and the EDT will be generated by decrypting the CT. The four images (PT, PDT, CT, CDT) will then be compared using various benchmarks to evaluate the quality and security of each homomorphic encryption scheme.

The benchmarks to be used in the study, adopted from \cite{ahmed_benchmark_2016, ahmad_efficiency_2012, wu_npcr_2011} are listed below. We let $X_i$ denote a value in an image $X$, where $1 \leq i \leq N$.

We first perform three tests to ascertain the preservation of image quality after encryption and decryption: MSE, PSNR, and SSIM.
\begin{description}
	\item [Mean Squared Error (MSE).] The MSE is defined \cite{ahmed_benchmark_2016} as
	\begin{align}
		MSE = \frac{1}{N}\sum_{i=1}^{N}{(CDT_i - PDT_i)^2}.
	\end{align}
	The MSE provides a measure of how much data is recovered if an image operation is applied on the encrypted image, which is then decrypted. Lower values of MSE indicate higher preservation of image quality \cite{ahmed_benchmark_2016, ahmad_efficiency_2012}.
	\item [Peak Signal to Noise Ratio (PSNR).]
	According to Ahmed, et al. \cite{ahmed_benchmark_2016}, PSNR is ``an estimator for human visual perception of reconstruction quality'' which is based on  It has been used to ascertain image quality in various studies and is a known meetric for image and video quality \cite{upmanyu_efficient_2009, ahmed_benchmark_2016, akramullah_video_2014}. Although it may produce results which do not correlate with human visual perception \cite{huynh-thu_accuracy_2012, ahmed_benchmark_2016}, it is a valid indicator of image quality when media containing the same visual content is compared \cite{huynh-thu_accuracy_2012}.
	PSNR is defined by
	\begin{align}
		PSNR = 10\log_{10}{\left( \frac{L^2}{MSE} \right)}
	\end{align}
	where $L$ is the maximum pixel intensity value of an image.
	Despite the known limitations of PSNR, since we are going to compare the effect of each encryption scheme on recovered image quality, given a fixed library of images, it is a valid measure of image quality for the study.
	\item [Structural Similarity Index (SSIM)]
	The SSIM for two random variables $X$ and $Y$ is defined \cite{ahmed_benchmark_2016, akramullah_video_2014} as
	\begin{align}
		SSIM(X,Y) = \frac{(2\mu_X\mu_Y+c_1)(2\sigma_{XY}+c_2)}{(\mu_X^2+\mu_Y^2+c_1)(\mu_X^2+\mu_Y^2+c_2)}
	\end{align}
	where
	\begin{itemize}
		\item $\mu_X, \mu_Y$ are the averages of $X$ and $Y$, respectively;
		\item $\sigma_X, \sigma_Y$ are the variances of $X$ and $Y$, respectively;
		\item $\sigma_{XY}$ is the covariance of $X$ and $Y$;
		\item $c_1 = (k_1L)^2, c_2 = (k_2L)^2$ are two variables used to stabilize the measure when $\mu_X^2+\mu_Y^2$ is close to zero \cite{akramullah_video_2014};
		\item $L$ is the the maximum pixel intensity value of an image;
		\item $k_1 = 0.01, k_2 = 0.03$ by default, given in \cite{ahmed_benchmark_2016}.
	\end{itemize}
	The SSIM is applied to the luminance value of two images to gauge structural similarity between neighboring pixels.
	For the study, we will compute $SSIM(PDT, CDT)$ for every image operation, under each homomorphic cryptosystem.
\end{description}

After performing tests to evaluate image quality, we then perform tests for cryptographic security, as given in \cite{ahmed_benchmark_2016}: entropy analysis, correlation coefficient analysis, NPCR and UACI.
\begin{description}
	\item [Information entropy analysis]
	\item [Correlation coefficient analysis]
	\item [Number of Pixel Change Rate (NPCR)]
	\item [Universeral Average Change Intensity (UACI)]
\end{description}

\section{Summary and Timetable}

\chapter{RESULTS AND DISCUSSION}
% After implementing your methodology and gathering all pertinent data, in this section, you will now present the gathered data to your reader. By the end of this section, your reader should have an idea of what exactlty happened during the experiment. 
% A good way to organize your results is to group them is to present them in the same order which your methodology was presented. For instance, if your methodology included the analysis of user logs, the implementation of an application, and the testing of this application, your results should flow in the same way. In addition, more often than not, you will be presenting a large volume of data, so utilize figures and tables whenever appropriate. Table 5.1 below presents one way of how to go about presenting your data. Note the table caption and headers, as mentioned in our framework.

% \begin{table}[t]
% \centering
% \caption {Preliminary Test Result, organized by Problem Type}
% \label{tab:acc_lit} 
% \begin{tabular}{ c  c  c  c  c  c  c  c  c  c  c }
% \hline
% Problem	&Average 	&Standard 	&Average 		&Standard		&Dominant\\
% Type  &Steps   &Deviation   &Duration   &Deviation   &Affective\\
%   &   &(Steps)   &(s)   &(Duration)   &State\\ \hline

% A1  &14   &2.30   &23.04   &3.50   &CONF\\
% A2  &2   &5.36   &32.10   &2.01   &FLOW\\
% A3  &31   &1.01   &28.55   &4.03   &FLOW\\
% B1  &24   &4.40   &45.30   &3.30   &BOR\\
% B2  &33   &2.12   &20.56   &2.21   &FLOW\\
% B3  &36   &1.05   &LOSE   &1.15   &CONF\\
% C1  &22   &1.33   &LOSE   &1.40   &FLOW\\
% C2  &23   &3.03   &LOSE   &1.30   &FLOW\\
% D1  &30   &1.79   &LOSE   &1.45   &FLOW\\
% D2  &15   &1.30   &LOSE   &1.05   &FLOW\\ \hline
% \end{tabular}
% \end{table}


% There are, however, some additional notes that must be clarified. First, given that you will be gathering a huge volume of data, you must be able to classify which of these were critical in determining the outcome of your study, and which ones need not be presented. The critical data must be presented in this section, while the minor ones may be placed in the Appendices of your paper, which will be described later in this template.

% Another clarification to be noted is that the presentation of results in this section must be objective, or `as-is'. This means that you must describe your results in a way understandable to your reader without putting any form of interpretation. In effect, this section’s intent is to provide answers to ``what happened'' questions, not ``what does it mean'' questions. The interpretation of results is the subject of a later section.

% Finally, because this is a presentation of what happened in the past, all tenses used in this section must be in the past form, be it active or passive. This will also be true for the preceeding sections after the study’s implementation, especially when stating the methodology.

\begin{table}[t]
\centering
\caption {Preliminary Test Result, organized by Problem Type}
\label{tab:acc_lit} 
\begin{tabular}{ c  c  c  c  c  c  c  c  c  c  c }
\hline
Problem	&Average 	&Standard 	&Average 		&Standard		&Dominant\\
Type  &Steps   &Deviation   &Duration   &Deviation   &Affective\\
  &   &(Steps)   &(s)   &(Duration)   &State\\ \hline

A1  &14   &2.30   &23.04   &3.50   &CONF\\
A2  &2   &5.36   &32.10   &2.01   &FLOW\\
A3  &31   &1.01   &28.55   &4.03   &FLOW\\
B1  &24   &4.40   &45.30   &3.30   &BOR\\
B2  &33   &2.12   &20.56   &2.21   &FLOW\\
B3  &36   &1.05   &LOSE   &1.15   &CONF\\
C1  &22   &1.33   &LOSE   &1.40   &FLOW\\
C2  &23   &3.03   &LOSE   &1.30   &FLOW\\
D1  &30   &1.79   &LOSE   &1.45   &FLOW\\
D2  &15   &1.30   &LOSE   &1.05   &FLOW\\ \hline
\end{tabular}
\end{table}

\section{Conclusion}
In this paper, we have assessed the feasibility and practicality of various homomorphic cryptosystems with regard to secure processing and manipulation of facial image data through creation of a software library that facilitates image processing operations within the encrypted domain. We have also extended the Paillier and DGK cryptosystems to support limited floating-point arithmetic, demonstrated in the implementation of the logarithm and power-law image intensity transformations. 

Preliminary results have shown that both Paillier and DGK apply image negation accurately and efficiently, as evidenced by the perfect zero MSE and infinite PSNR. For the logarithm transformation, Paillier was faster and accurate enough (with $\text{MSE} \le 30$ and $\text{SSIM} \approx 0.99$) which was rather unexpected. However both Paillier and DGK fail to produce accurate results in applying the power-law transformation, with the latter failing to produce discernible results. Based from these results, the Paillier cryptosystem is consistent enough to produce reasonable to accurate results.

\subsection{Ongoing and Future Work}
Testing of intensity transformations under the BGV cryptosystem will be started once the Pyfhel library is deemed stable enough to use, or until a similar library becomes available.
One of the authors of this paper has already informed the developers of the Pyfhel library about a bug involving exponentiation of two real numbers.

Efforts are ongoing with regard to the facial recognition tests. Our own implementation of the privacy-preserving eigenfaces by Erkin, et al. does not yet use the DGK cryptosystem for secure integer comparison in the match finding step.
Another promising image operation to consider is \textit{facial detection} under a homomorphic cryptosystem. Work has been started in order to assess the feasibility of using Haar cascades in a privacy-preserving manner.

Future work would involve improving and optimizing our implementation of non-linear intensity transformations in terms of number of operations, numerical accuracy and stability, and time efficiency. An implementation of a client-server system dealing with secure image processing can be considered.


\begin{acks}
% We would like to thank
% We would like to thank our adviser Dr. William Emmanuel S. Yu for his guidance and constant feedback, and
% our panelists Dr. Andrei D. Coronel and Dr. Mari-Jo P. Ruiz for lending their expertise and for giving their invaluable comments and insights.

Portions of this paper originally appeared as a chapter in the appendix of the first two authors' undergraduate thesis.
\end{acks}

\bibliographystyle{ACM-Reference-Format}
\bibliography{thesis_refs}

% \appendix
% \section{Correction for Dasgupta--Pal}
\label{chap:correction}
\subsection{Bitwise Description of the Dasgupta--Pal Cryptosystem}
The Dasgupta--Pal cryptosystem \cite{dasgupta_design_2016} encrypts each bit in a bit string independently.
In terms of the encryption of a single bit, the Dasgupta--Pal cryptosystem can be described as follows:
\begin{description}
	\item[Key generation]
	Let the secret key, $S_k$ be a large prime.
	Let the public refresh key, $R_k$, be $S_k \times z$, where $z$ is a large even integer.
	\item[Encryption]
	Given a message $m$, the encryption function is defined as
	\begin{align*}
		E(m) = m + S_kr
	\end{align*}
	where $r$ is a random integer.
	\item[Decryption]
	Given a ciphertext $c$, the decryption function is defined as
	\begin{align*}
		D(c) = c \bmod S_k \bmod 2
	\end{align*}
\end{description}

The homomorphic operations are then defined as
\begin{description}
	\item[\textsc{xor} on ciphertexts]
	\begin{align*}
		D(E(a)+E(b)) = a \textsc{ xor } b = (a + b) \bmod 2
	\end{align*}
	\item[\textsc{and} on ciphertexts]
	\begin{align*}
		D(E(a)\times E(b)) = a \textsc{ and } b = ab \bmod 2
	\end{align*}
\end{description}

After repeated operations on ciphertexts, the resulting ciphertext can grow in magnitude.
To prevent storage issues and keep the ciphertext size small, a refresh function is used.
\begin{description}
	\item[Refresh function]
	Given a ciphertext $c$, the refresh function is defined as
	\begin{align*}
		R(c) = c \bmod R_k
	\end{align*}
\end{description}
\subsection{Cases Where Decryption Fails}
We consider the following quantity:
\begin{align}
	\label{eq:cornercase_ciphertext}
	D(\underbrace{E(1)+E(1)+\cdots+E(1)}_{S_k \text{ times}}).
\end{align}

Since $S_k$ is odd, the corresponding operation in the plaintext space is
\begin{align*}
	\label{eq:cornercase_plaintext}
	\underbrace{1 \textsc{ xor } 1 \textsc{ xor } \cdots \textsc{ xor } 1}_{S_k \text{ times}} = 1.
\end{align*}

However, evaluating Equation \ref{eq:cornercase_ciphertext} yields:
\begin{align*}
	D(\underbrace{E(1)+E(1)+\cdots+E(1)}_{S_k \text{ times}})
	&= D\left(\sum_{i=1}^{S_k}{(1+S_kr_i)}\right)\\
	&= D\left(S_k + S_k\sum_{i=1}^{S_k}{r_i}\right)\\
	&= \left(S_k + S_k\sum_{i=1}^{S_k}{r_i}\right) \bmod S_k \bmod 2\\
	&= 0 \bmod 2\\
	&= 0.
\end{align*}
Since $\underbrace{1 \textsc{ xor } 1 \textsc{ xor } \cdots \textsc{ xor } 1}_{S_k \text{ times}} = 1\neq 0 = D(\underbrace{E(1)+E(1)+\cdots+E(1)}_{S_k \text{ times}})$, we have shown an instance where incorrect decryption occurs given a sequence of operations on ciphertexts.

\subsection{Proposed Correction and Proof of Correctness}
We now show that setting the secret key $S_k$ to $2p$, where $p$ is a prime, fixes the issue in the original cryptosystem.
\begin{theorem}
	Suppose $S_k = 2p$, where $p$ is a prime. Then when addition or multiplication is performed between any two ciphertexts, correct decryption is assured.
\end{theorem}
\begin{proof}
	Suppose $c_1, c_2$ are two ciphertexts such that
	\begin{align*}
		c_1 = a + S_kr_1\\
		c_2 = b + S_kr_2
	\end{align*}
	We first consider addition. We want to show that $(c_1+c_2)\bmod S_k$ has the same parity as $a+b$.
	By the division algorithm, we have $a+b = S_kq + r$, for some $q,r$, $q \geq 0, r > 0$. We can rewrite $c_1+c_2$ as
	\begin{align*}
		c_1+c_2 &= (a + S_kr_1) + (b + S_kr_2)\\
		&= (a+b)+ S_k(r_1 + r_2)\\
		&= (S_kq + r) + S_k(r_1 + r_2)\\
		&= r + S_k(q + r_1 + r_2).
	\end{align*}
	Therefore $(c_1+c_2)\bmod S_k = r$.
	It is enough to show that the parity of $a+b$ is the same as the parity of $r$. We consider cases based on the parity of $a+b$.
	\begin{description}
		\item[Case 1: $a+b$ is odd.]
			Since $a+b = S_kq + r$, $S_kq + r$ must also be odd.
			$S_k$ is even, so $S_kq$ is also even.
			For $S_kq + r$ to be odd, $r$ must be odd, since $S_k$ is even.

			Therefore, the parity of $a+b$ is the same as the parity of $r$.
		\item[Case 2: $a+b$ is even.]
			Since $a+b = S_kq + r$, $S_kq + r$ must also be even.
			$S_k$ is even, so $S_kq$ is also even. For $S_kq + r$ to be even, $r$ must be even , since $S_k$ is even.

			Therefore, the parity of $a+b$ is the same as the parity of $r$.
	\end{description}
	Therefore correct decryption is assured for the sum of ciphertexts.

	We now consider multiplication. We want to show that $c_1c_2 \bmod S_k$ has the same parity as $ab$. Similar to the addition case, by the division algorithm, we can write $ab = S_kq + r$, $q > 0, r > 0$ for some $q,r$. We can rewrite $c_1c_2$ as
	\begin{align*}
		c_1c_2 &= (a + S_kr_1) \times (b + S_kr_2)\\
		&= ab + S_k(br_1 + ar_2 + S_kr_1r_2)\\
		&= (S_kq + r) + S_k(br_1 + ar_2 + S_kr_1r_2)\\
		&= r + S_k(q + br_1 + ar_2 + S_kr_1r_2).
	\end{align*}
	Therefore $c_1c_2 \bmod S_k = r$.
	It is enough to show that the parity of $ab$ is the same as the parity of $r$. We consider cases based on the parity of $ab$.
	\begin{description}
		\item[Case 1: $ab$ is odd.]
			Since $ab = S_kq + r$, $S_kq + r$ must also be odd.
			$S_k$ is even, so $S_kq$ is also even.
			For $S_kq + r$ to be odd, $r$ must be odd, since $S_k$ is even.

			Therefore, the parity of $a+b$ is the same as the parity of $r$.
		\item[Case 2: $ab$ is even.]
			Since $ab = S_kq + r$, $S_kq + r$ must also be even.
			$S_k$ is even, so $S_kq$ is also even. For $S_kq + r$ to be even, $r$ must be even, since $S_k$ is even.

			Therefore, the parity of $ab$ is the same as the parity of $r$.
	\end{description}
	Therefore, correct decryption is assured for both addition and multiplication of ciphertexts.
\end{proof}



\section{Numerical Approximations}
\newcommand*\diff{\mathop{}\!\mathrm{d}}

Transcendental functions such as the exponential and logarithmic functions are usually implemented in computer hardware and software libraries using minimax polynomials, which are determined numerically using the Remez algorithm \cite{harrison_computation_1999}.
However, the Remez algorithm relies on iteratively refining the polynomial coefficients, which requires knowledge of the argument passed to the transcendental function.

We cannot directly apply this approach in privacy-preserving image processing as we do not have knowledge of the exact value of the function arguments. In order to calculate a function under a  homomorphic cryptosystem, it is necessary to express the function in terms of the homomorphic operations.
In this section, we discuss approximations for the logarithm ($\log(1+x)$) and power ($x^r$) functions using only addition, subtraction, multiplication, and division operations.

\subsection{Approximation for $\log(1+x)$}
\label{sec:logapproximation}
We approximate the function $f(x)=\log(1+x)$ using a similar method to that described in
\cite{khattri_new_2009}.
We let $x = 1/n$ and consider the integral
\begin{equation}
	\label{eq:log_integral}
  	\int_{n}^{n+1}{\frac{1}{x}\diff x}=\log{\left(1+\frac{1}{n}\right)}.
\end{equation}

This integral can be approximated using the five-point Gauss--Legendre quadrature rule \cite{kythe_quadrature_2002}. We first convert the integral to an integral over the interval $[-1,1]$ using the following transformation:
\begin{align*}
	\int_a^b{f(x)\diff x}
	&= \frac{b-a}{2}\int_{-1}^{1}{f\left(\frac{b-a}{2}x+\frac{a+b}{2}\right)\diff x}.
\end{align*}
Then, we approximate the integral using the following summation:
\begin{align*}
  \int_{-1}^{1}{f(x)\diff x} &= \sum_{i=1}^{5}{w_if(x_i)},
\end{align*}
where
\begin{multicols}{2}
	\noindent
	\begin{align*}
		w_1 &= 0\\
		w_2 &= \frac{1}{21}\sqrt{245-14\sqrt{70}}\\
		w_3 &= -\frac{1}{21}\sqrt{245-14\sqrt{70}}\\
		w_4 &= \frac{1}{21}\sqrt{245+14\sqrt{70}}\\
		w_5 &= -\frac{1}{21}\sqrt{245+14\sqrt{70}}
	\end{align*}
	\columnbreak
	\begin{align*}
		x_1 &= \frac{128}{225}\\
		x_2 &= \frac{1}{900}\left( 322 + 13\sqrt{70}\right)\\
		x_3 &= \frac{1}{900}\left( 322 + 13\sqrt{70}\right)\\
		x_4 &= \frac{1}{900}\left( 322 - 13\sqrt{70}\right)\\
		x_5 &= \frac{1}{900}\left( 322 - 13\sqrt{70}\right)
	\end{align*}
\end{multicols}
Applying this procedure to the integral in equation \ref{eq:log_integral} using SageMath 8.3 yields the following approximation:
\begin{equation}\label{eq:standardquadrature}
  \log(1+x) \approx
  \frac{137x^5 + 2310x^4 + 9870x^3 + 15120x^2 + 7560x}
  {30x^5 + 900x^4 + 6300x^3 + 16800x^2 + 18900x + 7560}.
\end{equation}

While the closed form approximation in equation \ref{eq:standardquadrature} is accurate for values of $x$ near zero, it diverges from $\log{(1+x)}$ significantly for large values of $x$, as shown in figure \ref{fig:standardquadrature}.
\begin{figure}[!ht]
    \centering
    \includegraphics[width=.8\linewidth]{figures/StandardQuadrature.png}
    \caption{Graph of $\log{(1+x)}$ and the approximation in equation \ref{eq:standardquadrature}}
    \label{fig:standardquadrature}
\end{figure}

As we only need accuracy for arguments $x \in [0, 255]$, we can scale the approximation by a constant factor $\alpha$ as follows:
\begin{align*}
  \log{(1+x)} &= \log{\left(\frac{\alpha + \alpha x}{\alpha}\right)}\\
  &= \log{(\alpha + \alpha x)} - \log{\alpha}\\
  &= \log{\left(\alpha+\frac{\alpha}{n}\right)} - \log{\alpha}\\
  &= \log{\left(\frac{\alpha n + \alpha}{n}\right)} - \log{\alpha}\\
  &= \int_{n}^{\alpha n + \alpha}{\frac{1}{x}\diff x} - \log{\alpha}
\end{align*}

Applying the five-point Gauss--Legendre quadrature rule with $\alpha = 1/20$ using SageMath 8.3, we arrive at the approximation:
\begin{align}\label{eq:scaledquadrature}
  \begin{split}
    &\log(1+x) \\
    &\approx \frac{137x^5 + 33185x^4 + 931370x^3 - 13403630x^2 - 289469315x - 713567363}
    {30(x^5 + 505x^4 + 42010x^3 + 923010x^2 + 5722005x + 8040501)} \\
    &+ \log{20}
  \end{split}
\end{align}

Figure \ref{fig:scaledquadrature} is a graph of the absolute error of the scaled approximation and the exact value of $\log{(1+x)}$. Using SageMath 8.3, it was numerically determined that the maximum absolute error of this approximation in the range $x\in[0,255]$ is approximately $0.0103315865985758$, occurring at $x=255$. This is an improvement from the approximation in equation \ref{eq:standardquadrature}, which has a maximum absolute error of $1.19717868468392$, similarly occurring at $x=255$.

\begin{figure}[h]
    \centering
    \includegraphics[width=.9\linewidth]{figures/ModifiedQuadratureAbsoluteError.png}
    \caption{Graph of the absolute error of $\log{(1+x)}$ and the approximation in equation \ref{eq:scaledquadrature}}
    \label{fig:scaledquadrature}
\end{figure}

\subsection{Approximation for $x^\gamma$}
To approximate $x^\gamma$ for any $\gamma \in \mathbb{R}$, we rewrite $x^\gamma$ as follows:
\begin{align*}
  x^\gamma = e^{\log{x^\gamma}} = e^{\gamma\log{x}}.
\end{align*}

This expression can then be approximated using the Maclaurin series expansion for $e^x$, which converges for all $x$.
\begin{align*}
  e^x &= \sum_{n=0}^{\infty}{\frac{x^n}{n!}}\\
  \Rightarrow e^{\gamma\log{x}} &= \sum_{n=0}^{\infty}{\frac{(\gamma\log{x})^n}{n!}}
\end{align*}

As we already have an approximation for the natural logarithm, we can evaluate partial sums of the above infinite series to arrive at approximations for $x^\gamma$.
% \balancecolumns

\end{document}
